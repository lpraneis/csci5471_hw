\documentclass[11pt]{article}
\usepackage{fullpage}
\usepackage{amsmath}
\usepackage{url}
\usepackage{hyperref}
\usepackage{graphicx}
\usepackage{amsfonts}
\newcounter{qnum}
\newcommand{\question}[1]{\stepcounter{qnum}\bigskip\noindent{\bf \arabic{qnum}. #1.}}

\newcommand{\E}{\mathop{\textrm{E}}}
\newcommand{\cM}{\mathcal{M}}
\newcommand{\cA}{\mathcal{A}}
\newcommand{\bbM}{\mathbb{M}}
\newcommand{\bbK}{\mathbb{K}}
\newcommand{\enc}{\textsf{Enc}}
\newcommand{\dec}{\textsf{Dec}}
\newcommand{\gen}{\textsf{Gen}}
\newcommand\NetIDa{prane001}          

\begin{document}
\begin{center}
{\Large \bf CSci 5471: Modern Cryptography}
\end{center}
{\bf Homework 4} \hfill {\bf due: April 19, 2019}
\medskip
\\
{\bf \NetIDa} \hfill  
\hrule
\medskip
\question{Number and Group Theory II} [30 points]  
\begin{itemize} 
  \item[(a)] Since $p = rq+1$, we know $p \equiv 1 \pmod{q}$ as $p-1 \equiv rq$. We will prove
    that $ord(g) | q$ where $g := x^{r}$. There are 2 cases: 
    \begin{itemize}
      \item Case 1: $g = x^r$ is a generator of the group. In this case, $g$ has order $q$.
        $$g^q \equiv x^{rq} \equiv x^{p-1} \equiv 1 \pmod{p}$$. 
      \item Case 2: $g$ is congruent to $1 \pmod{p}$. In this case, $g \equiv 1 \pmod{p}$ and so
        $ord(g) = 1 \rightarrow 1 | q$. 
    \end{itemize}
    In either case, $ord(g) = ord(x^r) | q$.
  \item[(b)] We can prove that $x \in\mathbb{Z}_p^*$, $\left| \{ y \in \mathbb{Z}_p^* 
    | y^r \equiv x^r \pmod p \} \right| = r$. We have that $(x^i)^r \equiv (x^j)^r \iff 
    ir \equiv jr \pmod{p-1}$ as $ord(x^r) | q.$ This is equivalent to $p -1 | (i -j) r$ and so
    $p -1 = rq$ implies $q|(i-j)$. Thus for $j \in \{0, 1, \cdots, p-1\}$, the set $i$ for which
    $(x^i)^r \equiv (x^j)^r$ is equivalent to $\{j, j+q, j+2q, \cdots, j+(r-1)q\}$ mod $p-1$. Thus
    there are $r$ values for this set, proving the function is r-to-1.
  \item[(c)] In order to prove that $\Lambda(B)$ is a group, we must prove 4 properties.
    \begin{itemize}
      \item Closure: For $B', B'' \in \Lambda(B)$, $B' + B'' \in \Lambda(B)$ is true as  
        $a_i' + a_i''$ for $x_i \in \mathbb{Z}$ and thus $\sum(a_i' + a_i'') \subset \Lambda(B)$
      \item Identity: True as $a_i = 0$ is the identity
      \item Inverse: True as $\tilde{a_i} = -a_i$, thus $a_i + \tilde{a_i} = 0$ for $i$ in two 
        vectors.
      \item Associative: This comes naturally from addition.
    \end{itemize}
    Thus, $\Lambda(B)$ is a group.
\end{itemize}
\question{Diffie-Hellman and Discrete Logarithms} [30 points]
\begin{itemize}
\item[(a)] % TODO 2a
\item[(b)] % TODO 2b
\item[(e)] % TODO 2e
\end{itemize}


\question{Number Theory II} [15 points]
\begin{itemize}
  \item[(a)] % TODO 3a
  \item[(b)] % TODO 3b
  \item[(c)] % TODO 3c
  \item[(d)] % TODO 3d

\end{itemize}

\question{RSA and factoring} [25 points] 
\begin{itemize}
  \item[(a)] % TODO 4a
  \item[(b)] % TODO 4b
  \item[(c)] % TODO 4c
\end{itemize}

\end{document}
