\documentclass[11pt]{article}
\usepackage{fullpage}
\usepackage{seqsplit}
\usepackage{amsmath}
\usepackage{url}
\usepackage{hyperref}
\usepackage{graphicx}
\usepackage{amsfonts}
\newcounter{qnum}
\newcommand{\question}[1]{\stepcounter{qnum}\bigskip\noindent{\bf \arabic{qnum}. #1.}}

\newcommand{\E}{\mathop{\textrm{E}}}
\newcommand{\cM}{\mathcal{M}}
\newcommand{\cA}{\mathcal{A}}
\newcommand{\bbM}{\mathbb{M}}
\newcommand{\bbK}{\mathbb{K}}
\newcommand{\enc}{\textsf{Enc}}
\newcommand{\dec}{\textsf{Dec}}
\newcommand{\gen}{\textsf{Gen}}
\newcommand\NetIDa{prane001}          

\begin{document}
\begin{center}
{\Large \bf CSci 5471: Modern Cryptography}
\end{center}
{\bf Homework 4} \hfill {\bf due: April 19, 2019}
\medskip
\\
{\bf \NetIDa} \hfill  
\hrule
\medskip
\question{Number and Group Theory II} [30 points]  
\begin{itemize} 
  \item[(a)] Since $p = rq+1$, we know $p \equiv 1 \pmod{q}$ as $p-1 \equiv rq$. We will prove
    that $ord(g) | q$ where $g := x^{r}$. There are 2 cases: 
    \begin{itemize}
      \item Case 1: $g = x^r$ is a generator of the group. In this case, $g$ has order $q$.
        $$g^q \equiv x^{rq} \equiv x^{p-1} \equiv 1 \pmod{p}$$. 
      \item Case 2: $g$ is congruent to $1 \pmod{p}$. In this case, $g \equiv 1 \pmod{p}$ and so
        $ord(g) = 1 \rightarrow 1 | q$. 
    \end{itemize}
    In either case, $ord(g) = ord(x^r) | q$.
  \item[(b)] We can prove that $x \in\mathbb{Z}_p^*$, $\left| \{ y \in \mathbb{Z}_p^* 
    | y^r \equiv x^r \pmod p \} \right| = r$. We have that $(x^i)^r \equiv (x^j)^r \iff 
    ir \equiv jr \pmod{p-1}$ as $ord(x^r) | q.$ This is equivalent to $p -1 | (i -j) r$ and so
    $p -1 = rq$ implies $q|(i-j)$. Thus for $j \in \{0, 1, \cdots, p-1\}$, the set $i$ for which
    $(x^i)^r \equiv (x^j)^r$ is equivalent to $\{j, j+q, j+2q, \cdots, j+(r-1)q\}$ mod $p-1$. Thus
    there are $r$ values for this set, proving the function is r-to-1.
  \item[(c)] In order to prove that $\Lambda(B)$ is a group, we must prove 4 properties.
    \begin{itemize}
      \item Closure: For $B', B'' \in \Lambda(B)$, $B' + B'' \in \Lambda(B)$ is true as  
        $a_i' + a_i''$ for $x_i \in \mathbb{Z}$ and thus $\sum(a_i' + a_i'') \subset \Lambda(B)$
      \item Identity: True as $a_i = 0$ is the identity
      \item Inverse: True as $\tilde{a_i} = -a_i$, thus $a_i + \tilde{a_i} = 0$ for $i$ in two 
        vectors.
      \item Associative: This comes naturally from addition.
    \end{itemize}
    Thus, $\Lambda(B)$ is a group.
\end{itemize}
\question{Diffie-Hellman and Discrete Logarithms} [30 points]
\begin{itemize}
  \item[(a)] We can factor $2^{32} -1 = 2^1 * 3^2 * 7^1 * 11^1 * 31^1 * 151^1 * 331^3$ and use
    the fact that if $g^{p / e_i} \not\equiv 1 \pmod{p} ,$ $g$ generates $\mathbb{Z}_p$. Thus 
    we can test $$
    7^{(2^{32}-1) / 2}, 
    7^{(2^{32}-1) / 3},
    7^{(2^{32}-1) / 7},
    7^{(2^{32}-1) / 11},
    7^{(2^{32}-1) / 31},
    7^{(2^{32}-1) / 151},
    7^{(2^{32}-1) / 331}$$
    And see that they all are not equivalent to 1 mod p,  proving that $ord(7) = p-1$, and 
    thus, 7 generates the group.
\item[(b)] Code submission linked. Baby step giant step algorithm was used to compute the 
  discrete log
\item[(e)] 
  \begin{itemize}
    \item[(i)] In order to compute a ciphertext $c_3$ from $c_1, c_2$, we can simply multiply the
      ciphertexts, e.g.
      \begin{align}
        c_1&= \langle g^{y_1}, g^{xy_1}m_1 \rangle \nonumber \\
        c_2&= \langle g^{y_2}, g^{xy_2}m_2 \rangle \nonumber \\
        c_3&= \langle g^{y_1 + y_2}, g^{x(y_1 + y_2)}(m_1  m_2) \rangle \nonumber 
      \end{align}
    \item[(ii)]
      We can blind the plaintext in an Elgamal scheme so this is difficult to distinguish.
      In order to blind the ciphertext, let us pick a random $b \in_R \mathbb{Z}_{p-1}$, the 
      same way $x$ is chosen. Now we can compute $E(b)$ and $b^{-1} (\mathbb{Z}_{p-1})$. Using
      this information, we encrypt $m$ as $E(m)*E(b)$, to get $E(mb)$. Then, upon decryption, we
      simply use $D(mb) = mb * b^{-1} = m$ to recover the original message. This is a blind 
      encryption as $b \in_R \mathbb{Z}_{p-1}$ over DDH. Thus $\Pr[b=b*] = \frac{1}{2}$ if b is
      chosen uniformly.
  \end{itemize}
\end{itemize}


\question{Number Theory II} [15 points]
\begin{itemize}
  \item[(a)]  For a set of $n$ integers not exceeding $2n$, there must be 2 that are relatively 
    prime.   For $1 \leq x \leq n$, we have a set of these integers, $S = \{2x-1, 2x\}$ that 
    is coprime as $gcd(2x-1, 2x) =1$. For the range of $x$, $|S| = n$ and so with $n+1$ integers, 
    we must have at least coprime pair.
  \item[(b)] We can prove this inductively. We know $gcd(f_0, f_1) =1$ by observation. Now, we 
    suppose $gcd(f_n, f_{n+1}) =1$ for some $n$. We need to prove $gcd(f_{n+1}, f_{n+2})=1$. 
    By using the fact of additive fact of gcd, eg that:
    \begin{align}
      gcd(a,b) &= \alpha a + \beta b \nonumber \\
      gcd(a, a+b) &= \hat \alpha a + \hat \beta (a + b) \nonumber \\
                  &= (\hat \alpha + \hat \beta) a + \hat \beta b \nonumber \\
                  &= gcd(a,b) \nonumber
    \end{align}
  the inductive step becomes simple as 
  $gcd(f_{n+1}, f_{n+2}) = gcd(f_{n+1}, f_{n+1} + f_n) = gcd(f_{n+1}, f_n) = 1$ by the inductive
  hypothesis. Thus, the gcd of two neighboring terms of the Fibonacci sequence equals 1.
\item[(c)] Let $d = gcd(a,b).$ We know that $\alpha a + \beta b = kd$ for 
  $\alpha, \beta, k \in \mathbb{Z}$. 
  Thus, if $x,y$ exist, we know $c=kd$ and thus $d = gcd(a,b) | c$
  \item[(d)] We know there cannot exist a solution to this set of equations due to the following 
    sequence of equalities:
    \begin{align}
      5x \equiv 1 \pmod{6} &\rightarrow 6|(5x-1) \rightarrow 6|5(5x-1) \nonumber \\
                           &\rightarrow 6|(x-5) \rightarrow 3|(x-5) \nonumber \\
      x \equiv& 5 \pmod{3} \nonumber \\
      4x \equiv 13 \pmod{15} &\rightarrow 15|(4x-13) \rightarrow 15|4(4x-13) \nonumber \\
                           &\rightarrow 15|(x-7) \rightarrow 3|(x-7) \nonumber \\
        x \equiv& 7 \pmod{3} \nonumber 
    \end{align}
    Thus, since x cannot be congruent to both 5 and 7, there is no solution.
\end{itemize}

\question{RSA and factoring} [25 points] 
\begin{itemize}
  \item[(a)] 
  e =  0xf1bc3e139ef1b179306accf0809920a10e9f3b2a12065ee12fa7d6e519261431

  d = \seqsplit{
    0xd e891f2f 7a886ce afd86fe f69853a 2f1322c 111cf76 b83e6a2 20015d2 2507138 9c2f4c2 828f532 e381808 30e638d 584e
    7a4 3afac2e ee1eae6 a41cae9 5a07978 d58c6a2 3ef8b63 0e43ba3 d8ff6b8 49950ca acdaa82 9b4d6c8 a05edf1 a30b90f 2fcdb3
    149 deb6a2f 18286e9 70a7232 0ef9d5d f4803ed d2e5dbb ebf0ab9 a61b068 8d16175 38e4ff6 df8d955 e5ca13e 1e0dbb7 fec186
    f3a c1d3fc1 6865ea8 64d0bdf 654cfbf 7e7a71e 24bfad5 bb5d168 8ad7b88 3a32f1f caa60f1 f47696b dfc626c fff19d6 5aaed0
    c1a 336c9ed 94b033b 1c0491b ffede90 e2fd53d 4a92648 1463933 4dba524 6fa66cd a641f59 f4e9967 fb12ea4 085caec ee430c
    6d9 a623e7f 1dc6a}
  \item[(b)] 

    Plaintext:
    YEAH UT OH YOU FEEL THAT ALRIGHT COME ON DONT STOP NOW YOU DONE DID IT COME ON UH YEAH 
    ALRIGHT HOLD ON BABY WHEN WERE GRINDING I GET SO EXCITED OOH HOW I LIKE IT I TRY BUT I 
    CANT FIGHT IT OH YOURE DANCING REAL CLOSE CUZ ITS REAL REAL SLOW YOU KNOW WHAT YO

    File: 1166.txt

    Position: byte 86 - 342

    In order to recover the plaintext, I decoded the public key using pem decoding. Then, in a 
    loop, tried to encrypt every file in the LAPDOG database with this public key exponent, 
    stopping when a match with the ciphertext was found. I used ct7, pt7 for the last digit in
    my student ID.

  \item[(c)]  Since all of the public keys are known publicly, an attacker can utilize the 
    Chinese Remainder Theorem.  The modular equivalencies are shown below:
    \begin{align}
      x& := (r_A^3)^{\frac{1}{3}} = r_A  \nonumber \\
      x& \pmod{ N_B} \equiv y_1 \nonumber \\
      x& \pmod{ N_C} \equiv y_2  \nonumber \\
      x& \pmod{ N_D} \equiv y_3  \nonumber
    \end{align}
    Where $y_i$ is the $i^{th}$ part of the ciphertext sent that corresponds to the right
    public key, e.g. $ct = (y_1, y_2, y_3, c_A).$
    Using the theorem allows an attacker to find the $r_A$ and thus compute $H(r_A)$, 
    as $H$ is a public hash function. Then, the attacker will be able to find $m_A$ easily as $c_A$
    is sent in the ciphertext.
\end{itemize}

\end{document}
