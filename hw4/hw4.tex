\documentclass[11pt]{article}
\usepackage{fullpage}
\usepackage{amsmath}
\usepackage{url}
\usepackage{hyperref}
\usepackage{graphicx}
\usepackage{amsfonts}
\newcounter{qnum}
\newcommand{\question}[1]{\stepcounter{qnum}\bigskip\noindent{\bf \arabic{qnum}. #1.}}

\newcommand{\E}{\mathop{\textrm{E}}}
\newcommand{\cM}{\mathcal{M}}
\newcommand{\cA}{\mathcal{A}}
\newcommand{\bbM}{\mathbb{M}}
\newcommand{\bbK}{\mathbb{K}}
\newcommand{\enc}{\textsf{Enc}}
\newcommand{\dec}{\textsf{Dec}}
\newcommand{\gen}{\textsf{Gen}}

\begin{document}
\begin{center}
{\Large \bf CSci 5471: Modern Cryptography}
\end{center}
{\bf Homework 4} \hfill {\bf due: April 19, 2019}
\medskip
\hrule
\medskip
\noindent{\bf Ground Rules.} You may choose to complete these
homeworks in a group of up to three students.  Each group should turn
in {\bf one} copy with the names of all group members on it.  You may
use any source you can find to help with this assignment but you {\bf
  must} explicitly reference any source you use besides the lecture
notes or textbook.  Electronically typeset copies of your solution
should be submitted via canvas by 11:59\textsc{pm} on the
date above.

\question{Number and Group Theory II} [30 points]  
\begin{itemize} 
  \item[(a)] [10 points] Let $p$ and $q$ be primes such that $p = rq+1$ for some
    $r \in \mathbb{N}$.  Prove that for any $x \in \mathbb{Z}_p^*$,
    $ord(x^r) | q$.
  \item[(b)] [10 points] Let $p,q,r$ be as in part (a).  Prove that the
    map $x \mapsto x^r$ is $r$-to-1, i.e. for all $x \in
    \mathbb{Z}_p^*$, $\left| \{ y \in \mathbb{Z}_p^* | y^r \equiv x^r \pmod p
    \} \right| = r$. 
  \item[(c)] [10 points] Let $B = \{\vec{x}_1,\dots,\vec{x}_k\} \subset
    \mathbb{R}^n$ be a set of $n$-dimensional real vectors, and let
    $\Lambda(B) = \{ \sum_i a_i \vec{x}_i | a_i \in \mathbb{Z} \}$ denote
    the set of all integer linear combinations of $B$.  Prove that $\Lambda(B)$
    is a group under vector addition.
\end{itemize}
\question{Diffie-Hellman and Discrete Logarithms} [30 points]
\begin{itemize}
\item[(a)] [10 points] Let $p = 2^{31}-1$.  Prove that $7$ generates $\mathbb{Z}_p^*$.
\item[(b)] [10 points] Program an algorithm to compute discrete
  logarithms to base $7$ mod $p$.  Let $n = SHA256(\textrm{your CSELabs
    username})$. Use your algorithm to compute $\log_7 n \bmod p$.
  Include the code for your program as a separate submission to moodle. 

  The program \verb#/class/csci5471/hw4/dlgen.py# can be used to
  compute the value $n \bmod p$ (printed to the standard output in decimal) if
  you want to double-check your result.

\item[(c)] [Extra credit: 10 points] Compute $\log_7 n \bmod
  2^{80}-65$, or for 5 points compute $\log_7 n \bmod 2^{64}-59$.
  Explain any modifications used to make your algorithm complete in time.

\item[(e)] [10 points] Let $g^x$ be an Elgamal public key over prime $p=2q+1$,  and $c_1 =
  \langle g^{y_1}, g^{xy_1}m_1 \rangle$, $c_2 = \langle g^{y_2},
  g^{xy_2} m_2\rangle$ be a pair of ciphertexts encrypted with this
  public key.  Describe how to compute a ciphertext $c_3$ for $m_1m_2 \pmod p$
  without knowing either $m_1$ or $m_2$.  Show how to {\em blind} the
  resulting ciphertext so that given $c_1,c_2,c_3$ it is hard (under
  the decisional Diffie-Hellman assumption) to distinguish between the
  cases that $c_3 = E(m_1m_2)$ and $c_3 = E(m_3)$ where $m_3
  \not\equiv m_1m_2 \pmod p$.  (An encryption scheme with this
  property -- the ability to ``mutliply'' the plaintexts -- is called
  ``multiplicatively homomorphic'')

\item[(f)] [Extra credit: 15 points] Describe how to modify Elgamal
  encryption so that it is {\em additively homomorphic}, i.e. there is
  a method to ``sum'' the plaintexts of two ciphertexts.   In order to
  allow efficient decryption, you will need to restrict the space of
  plaintexts.  Implement your scheme; it should be able to efficiently
  handle 8-byte plaintexts.    Your implementation should support the
  following command-line options:
  \begin{itemize}
    \item \verb#genkey nbits pubfile privfile# generates a key pair of
      {\tt nbits} bits, writing the public key to {\tt pubfile} and
      the private key to {\tt privfile}.
    \item \verb#encrypt keyfile msg ctextfile#  encrypts up to 6 bytes of {\tt
        msg} under the public key found in {\tt keyfile} and writes
      the result to {\tt ctextfile}.
    \item \verb#decrypt keyfile ctextfile# decrypts the ciphertext
      stored in {\tt ctextfile} using the secret key in {\tt keyfile},
      assuming the plaintext is in $\{0,..,2^{48}\}$, and writes the
      plaintext to the standard output.

    \item \verb#add keyfile cfile1 cfile2 outfile# adds and blinds the
      ciphertexts stored in files {\tt cfile1} and {\tt cfile2},
      assuming they are encrypted under the public key stored in {\tt
        keyfile}.  The result should be written to {\tt outfile}.
  \end{itemize}
\end{itemize}


\question{Number Theory II} [15 points]
\begin{itemize}
  \item[(a)] Show that in any set of $n+1$ positive integers not exceeding
    $2n$, there must be two integers that are relatively prime. (Hint: Use
    the Pigeonhole Principle.) 
  \item[(b)] Define the Fibonacci sequence of integers as follows:
    $F_0 = 1, F_1 = 1$ and $F_{n+2} = F_{n+1}+F_n$ for $n \geq 0$. Show that
    $\gcd(F_i, F_{i+1}) = 1$ for all $i$.
  \item[(c)] Show that $ax + by = c$ has integer solution $(x, y)$ if
    $\gcd(a,b) ~|~ c$. 
  \item[(d)] [Extra Credit: 5 points] Determine whether the congruences $5x \equiv 1
    \pmod{6}$, $4 x \equiv 13 \pmod{15}$ have a common solution, and find it
    if they exist. 
\end{itemize}

\question{RSA and factoring} [25 points] 
\begin{itemize}
\item[(a)] [5 points] The files \verb#/class/csci5471/hw4/p# and
  \verb#/class/csci5471/hw4/q# contain 1024-bit prime numbers encoded
  using ASN.1 DER encoding.   Compute the decryption exponent
  corresponding to the encryption exponent represented by the SHA256
  hash of your CSELabs username.\footnote{If the hash of your username
    is even, add 1.}  Your main submission should give both
  numbers in hexadecimal, and any source code used should be included
  as separate files in your canvas submission.

\item[(b)] [10 points] In the CSELabs directory
  \verb#/class/csci5471/hw4/vanilla/# there are files labelled
  \verb#pk0#, \verb#pk1#, \dots, \verb#pk9# and \verb#ct0#,
  \verb#ct1#, \dots, \verb#ct9#.   The file \verb#ctN# contains a
  ciphertext encrypted under the 2048-bit RSA public key stored in \verb#pkN#.
  Public keys are encoded in PEM format (RFC 1421,
  \url{http://www.ietf.org/rfc/rfc1421.txt}).   
  Ciphertexts are encrypted using ``vanilla'' (a.k.a. ``textbook'') RSA, and
  each ciphertext encodes 256 bytes from one of the digital objects in the
  LAPDOGs database  (\verb#/class/csci5471/hw1/db/#).  Recover the
  plaintext for the file corresponding to the last digit of your
  student ID number.  Your main submission should include the
  recovered plaintext, and describe your approach to recovering the
  plaintext.  Include any source code used as separate files in your
  canvas submission.

\item[(c)] [10 points] Alice has a {\em very secret message} $m_A$ to send
  to Bob, Carol, and Dave, who all have 2048-bit RSA keys
  ($N_B$,$N_C$, and $N_D$, respectively) with public
  exponent $e=3$.  To encrypt her message, Alice chooses a random
  number $r_A \in_R \{0,1\}^{2032}$, computes a 256-bit symmetric key $K_A =
  H(r_A)$, computes $c_A = E_{K_A}(m_A)$, the encryption of $m_A$
  using AES-GCM, and sends $r_A^3 \bmod N_B, r_A^3 \bmod N_C, r_A^3
  \bmod N_D, c_A$ to all recipients.  Describe an attack that will
  allow anyone who knows the public keys $N_B,N_C,N_D$ to recover
  $m_A$.   (You may use the fact that computing integer cube roots is easy.)

\item[(d)] [Extra Credit: 10 points] The fastest general-purpose
  factoring algorithms all work by finding $x \not\equiv \pm y \pmod n$
  such that $x^2 \equiv y^2 \pmod n$.  For this problem, we'll
  implement the final step -- computing factors from $x$ and $y$.  The
  program \verb#/class/csci5471/hw4/sq#  has two command line
  options.  Running \verb#sq mod# prints a ASN.1 DER INTEGER encoding
  of a 2048-bit RSA modulus $n$ to the standard output (this has
  nonprinting characters, so you'll want to redirect it to a file).
  Running \verb#sq sqrt file# reads an ASN.1 DER INTEGER encoding of
  an integer $a$ from \verb#file# and prints (an ASN.1 DER INTEGER
  encoding of) an integer $y$ such that $y^2 \equiv a \pmod n$ to the
  standard output (this also has nonprinting characters).  Since
  half of the square roots of $a$ are not $\pm y \bmod n$, then
  choosing $x \in_R \mathbb{Z}_n^*$, and setting $a = x^2 \bmod n$
  gives a 50\% chance that $x \not\equiv \pm y \pmod n$.

  Each student has a unique modulus.  Find the factors of your modulus
  and include an explanation of your method in your main submission,
  along with any source code used as separate files.
\end{itemize}

\end{document}
