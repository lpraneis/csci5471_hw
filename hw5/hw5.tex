\documentclass[11pt]{article}
\usepackage{fullpage}
\usepackage{amsmath}
\usepackage{url}
\usepackage{hyperref}
\usepackage{graphicx}
\usepackage{amsfonts}
\newcounter{qnum}
\newcommand{\question}[1]{\stepcounter{qnum}\bigskip\noindent{\bf \arabic{qnum}. #1.}}

\newcommand{\E}{\mathop{\textrm{E}}}
\newcommand{\cM}{\mathcal{M}}
\newcommand{\cA}{\mathcal{A}}
\newcommand{\bbM}{\mathbb{M}}
\newcommand{\bbK}{\mathbb{K}}
\newcommand{\enc}{\textsf{Enc}}
\newcommand{\dec}{\textsf{Dec}}
\newcommand{\gen}{\textsf{Gen}}

\begin{document}
\begin{center}
{\Large \bf CSci 5471: Modern Cryptography}
\end{center}
{\bf Homework 5} \hfill {\bf due: May 3, 2019}
\medskip
\hrule
\medskip
\noindent{\bf Ground Rules.} You may choose to complete these
homeworks in a group of up to three students.  Each group should turn
in {\bf one} copy with the names of all group members on it.  You may
use any source you can find to help with this assignment but you {\bf
  must} explicitly reference any source you use besides the lecture
notes or textbook.  Electronically typeset copies of your solution
should be submitted via canvas by 11:59\textsc{pm} on the
date above.

\question{Proofs of RSA Signatures} [20 points]
Lecture 23 covers the Fiat-Shamir approach to constructing an
electronic signature scheme: first we design a (honest-verifier)
zero-knowledge proof for some hard problem; then we convert it to a
signature scheme by using an instance of the hard problem as a public
key, and sign a message $m$ by using the hash of $m$ and the prover's
message in place of the verifier message.
\begin{itemize}
\item[(a)] Design a three-round zero knowledge proof scheme that the Prover
  knows an RSA-FDH signature (using key $(N,e)$) on the message $m$.
  Prove that your scheme is sound and zero-knowledge.

  Your scheme should look similar to the zero-knowledge proof of knowledge for
  discrete logarithm.  In that proof, the prover either shows that they
  know the discrete logarithm of some $h \in \mathbb{G}$ or that $h$
  was formed  by taking $h = g^r y$.  In this case, note that we can
  show the knowledge of an RSA signature on a random element $r$ of
  $\mathbb{Z}_N^*$ by picking $s \in_R \mathbb{Z}_N^*$ and taking $r =
  s^e \bmod N$.

\item[(b)] Give a complete description of a signature scheme based on your
  zero-knowledge proof scheme from step (a).  Note that you may need
  many parallel repetitions of the basic zero-knowledge scheme in
  order to get a scheme that is hard to forge.

\item[(c)] [Extra Credit: 10 points] An {\em Identity-Based Signature Scheme} is a signature
  scheme in which a {\em Master Secret Key} $MSK$ is used to derive {\em
    identity signing keys} for some set $\mathcal{I}$ of identities.  An
  identity signing key can be used (along with a Master Public Key
  $MPK$) to produce signatures so that a verification function (which takes as input $MPK$,
  an identity $i \in \mathcal{I}$, a message $m \in \mathbb{M}$ and a
  signature $\sigma$ and verifies whether $(m,\sigma)$ is a valid
  signature under identity $i$).  The security goal of an IBS scheme
  is roughly that an adversary who knows $MPK$, identity signing keys $I \subset \mathcal{I}$,
  and signatures for a chosen identity $i \not\in I$ on a set of messages $M$, cannot
  produce a new, valid $(m,\sigma)$ pair for identity $i$.

  Show how to use your scheme from part (b) to create an IBS scheme,
  and argue for the security of your scheme.
\end{itemize}
\newpage
\question{Signature Substitution Attacks} [20points]
It is often argued that it is sufficient for ``hash-and-sign'' signatures to use
a 2nd-preimage resistant hash function, since a collision between two random
messages seems likely to be useless.  Let's explore this in the context of protocols that use trusted signatures.
\begin{itemize}

  \item[(a)] In a ``chosen-prefix'' collision attack, the adversary is given a
  pair of prefixes $P_1, P_2$ and asked to find strings $s_1 \neq s_2$ such that
  $H(P_1||s_1) = H(P_2||s_2)$.  Show that the generic birthday attack complexity
  analysis applies to chosen-prefix attacks (i.e., that they have expected
  complexity about $2^{k/2}$ for a $k$-bit hash.)

  Explain how an efficient chosen-prefix attack on the hash algorithm used by a
  Certificate Authority could be used to obtain fraudulent certificates, even if the CA is
  trustworthy.

  \item[(b)] Mobile operating systems like iOS and Android support the idea of code signing, in which a program is inspected by some trusted process and then signed, and only signed code can be installed.  In the directory \verb#/class/csci5471/hw5# you will find two files, \verb#coll1.py# and \verb#coll2.py# that have the same MD5 hash.  Use these files (and the fact that MD5 is a Merkle-Damg{\aa}rd hash) to write two programs, \verb#fluffybunny.py#, and \verb#eevilRat.py#, such that (i) for every input, \verb#fluffybunny# produces a completely innocent output (e.g. an ascii art bunny \verb#(o:)3 -($input)# ``saying'' the input string); (ii) \verb#eevilRat.py# ``spies'' on the user by appending its input to a file named \verb#pwn.txt#; and (iii) both files have the same MD5 hash.  (Thus a signature on \verb#fluffybunny# could be used as a proof that \verb#eevilRat# was safe to run.)  Creative approaches to hiding the behavior of your program(s) will be appreciated.
\end{itemize}

\question{Middleperson attacks} [20 points] All of the major Internet crypto
protocols try to defend against ``person in the middle attacks'' (in
which Alice ``securely'' talks to Eve and Eve ``securely'' talks to Bob, while Alice and Bob
believe they are communicating directly) using public key
certificates: digitally signed strings that bind an identity to an
encryption or verification key.  If the certificate ``authority'' (CA)
that signs these certificates is trustworthy, Alice and Bob can build
protocols around these certificates that ensure no third party can
intercept or modify their communications.  One potential problem here
is that the set of all CAs that most programs accept is not
particularly worthy of this sort of trust; let's look at some of the
other problems that can come up in this model.
\begin{itemize}
\item[(a)] Certificates usually include several ``attributes'' in
  addition to a name and public key.  The most security critical of
  these attributes is the ``CA'' boolean, which tells whether this
  certificate can be used to sign other certificates.  Having a signed
  certificate with this attribute is equivalent to being a root
  certificate authority, unless the certificate is revoked.  Another
  important attribute deals with revocation, stating where to
  find out whether the certificate has been revoked.

  Both of these attributes are optional: a certificate may be issued
  with no CA attribute and no revocation option.  Suppose that a CA-signed
  certificate has no CA attribute; how should an implementation
  evaluate a certificate ``chain'' in which this certificate has been
  used to sign another certificate?  Suppose that an implementation
  mistakenly makes the opposite interpretation; describe how to mount
  a middleperson attack on such an implementation.

\item[(b)] The most common format for certificates is the X.509 standard, described in
  \url{http://tools.ietf.org/html/rfc5280}.  Section 4.2 describes the
  common ``extensions'' that can be included as attributes.   Examine
  the list and explain how at least one other attribute that could be
  misinterpreted by an implementor, allowing a middleperson attack.

\item[(c)] [Extra Credit: 15 points] To illustrate the impact of parts
  (a) and (b), produce a pair of signed x.509v3 certificate signing requests
  (in PEM format) such that (i) the first request is for a ``benign'' subject and does
  not contain a ``basic-constraints'' extension field (so is not a
  valid CA certificate), (ii) the second request DOES contain a
  ``basic-constraints'' extension field that sets the cA field to
  TRUE, and (iii) the last 8 bytes of the SHA1 hash of both requests
  is the same.  You may find the ``subject alternative identifier''
  extension useful, since CAs are required to support it and it can
  contain essentially arbitrary short strings (in e.g. email or URI
  format).   Describe how you produced the files, and include them as
  separate files in your canvas submission.
\end{itemize}

\question{Forged in the fires of CSELabs\dots} [20 points]
\begin{itemize}
\item[(a)] [10 points] Many descriptions of the DSA signature algorithm emphasize
  the fact that the secret random value $k$ must not be used twice,
  since a pair of signatures with the same $k$ allow an adversary to
  extract the secret key $a$  (see HAC pp. 452--456 for the variable
  names used in this section).  Ivan the implementor decides that in
  order to avoid the possibility of a collision, his implementation
  will choose a random value $k \in_R \mathbb{Z}_q$ at key generation,
  and store this value with the secret key.  For each signature, the
  value $k$ will be incremented by $1$.  (This way, Alice can only
  repeat a value of $k$ if she signs $q > 2^{159}$ messages)

  Show how to use a pair of consecutive signatures to recover Alice's
  secret key $a$.

\item[(b)] [Extra Credit: 10 points] Someone points out your attack to
  Ivan, so he releases a patch: now at key generation, a random value
  $k \in_R \mathbb{Z}_q$ and a random offset $\rho \in_R \mathbb{Z}_p$
  are both stored with the secret key.  When signing a message, the
  stored value of $k$ is used and then is updated by setting $k \gets
  k + \rho \bmod q$.  Recall that $q$ is prime.  Show that (i) $k$
  will cycle through all of $\mathbb{Z}_q$ before repeating, and (ii)
  your attack can still be extended to this case (you may require more
  than 2 consecutive signatures).

\item[(c)] [5 points] Many internet protocols add ``authentication''
  by including a random, secret $k$-bit value in a request, which is
  then included in the reply.  For example, a DNS request has a 16-bit
  transaction identifier and a 16-bit source port which must match for
  a result to be accepted.  Show that even if the transaction identifier was
  encrypted, it would not defend against an adversary that can see and
  modify or inject messages between the client and the server.

\item[(d)] [5 points] HAC section 11.6.2 describes the Merkle
  signature scheme, and notes that it is important to encode the
  message $m$ as $m||c$, where $c$ is the number of 0-bits in $m$.
  Suppose we instead encode the message by $H(m)$, where $H$ is a
  $160$-bit cryptographic hash function suitable for modeling as a random
  oracle.  Show that this scheme is vulnerable to an existential
  forgery given the signature on a single chosen message.
\end{itemize}
\newpage
\question{Too Much (Private) Information} [20 points]
\begin{itemize}

\item[(a)] Suppose that Alice wants to build a PIR system allowing
  clients to privately retrieve the revocation status of certificates.
  This could be a problem because a certificate is not a number
  between $1$ and $n$ as in the basic PIR setting.  Describe how to
  extend the basic PIR setup to allow this type of query.

\item[(b)] Suppose Alice decides that (a) computational PIR is just
  too expensive and (b) there is no pair of servers that {\em someone}
  on the internet will not suspect of colluding.  How could she extend
  the basic two-server PIR scheme described in lecture to support $m$
  servers such that each client can pick a subset of size $k < m$ that
  she trusts not to mutually collude? (Clearly the communication
  complexity should still be smaller than the size of the database.)
  Sketch a proof that your scheme provides perfect query privacy.
\end{itemize}
\end{document}
