\documentclass[11pt]{article}
\usepackage{fullpage}
\usepackage{amsmath}
\usepackage{url}
\usepackage{hyperref}
\usepackage{graphicx}
\usepackage{amsfonts}
\newcounter{qnum}
\newcommand{\question}[1]{\stepcounter{qnum}\bigskip\noindent{\bf \arabic{qnum}. #1.}}

\newcommand{\E}{\mathop{\textrm{E}}}
\newcommand{\cM}{\mathcal{M}}
\newcommand{\cA}{\mathcal{A}}
\newcommand{\bbM}{\mathbb{M}}
\newcommand{\bbK}{\mathbb{K}}
\newcommand{\enc}{\textsf{Enc}}
\newcommand{\dec}{\textsf{Dec}}
\newcommand{\gen}{\textsf{Gen}}
\newcommand\NetIDa{prane001}          

\begin{document}
\begin{center}
{\Large \bf CSci 5471: Modern Cryptography}
\end{center}
{\bf Homework 5} \hfill {\bf due: May 3, 2019}
\medskip
\\
{\bf \NetIDa} \hfill  
\hrule
\medskip

\question{Proofs of RSA Signatures} [20 points]
\begin{itemize}
  \item[(a)] Let $P, V$ the prover and the verifier in the following 3 round protocol. Initially, 
  the prover generates $s\in_R \mathbb{Z}_N^*$ and sets $r = s^e \pmod{N}$. They want to prove to
  the verifier that they know a RSA-FDH signature, i.e. $\sigma = \textsf{H}(m)^d \pmod{N}$. 
  In order to do this, we can implement the following protocol:
  \begin{align}
    &P\rightarrow V: r  \\
    &V: \text{Generate } k\in_R \mathbb{Z}_N^* \nonumber \\
    &V\rightarrow P: k  \\
    &P\rightarrow V: k^ds  \\
    &V: \text{Accept if } (sk^d)^e \equiv kr \pmod{N}\nonumber
  \end{align}
  We can prove that this scheme is sound as $k$ acts as a \textit{blind}. Therefore, 
  if we have a system of $(P^*, V)$, the probability that $V$ accepts $(sk^d)^e \equiv kr \pmod{N}$
  is significantly $< \frac{1}{2}$, as the $P^*$ has no knowledge about the RSA signature. 
  Therefore, $P^*$ would choose $s, r$ relatively randomly, giving the probability of equivalence 
  around $\frac{1}{N}$. We can also prove that this scheme is Zero Knowledge. The Zero Knowledge 
  property requires that there exists a $S$ that can simulate a transcript between $(P, V^*)$. 
  This property comes naturally from the verifier's choice of $k$. In a simulator, the values
  of $r,k$ can be chosen, making the simulation of a transcript easy to actually verify. Thus, the
  scheme is ZK.
  \item[(b)] We can use the Fiat-Shamir Method to create a signing scheme from this ZK proof. 
    \begin{align}
      \text{Sign}(m):&  \nonumber & \\
                     & s_1 = r\in_R \mathbb{Z}_N \nonumber \\
                     & s_2 = H(r, m) \nonumber \\
                     & s_3 = (s_2)^{d}s \nonumber \\
                     & \text{Output } s_1, s_3 \nonumber \\
      \text{Verify} (s_1, H(s_1, m), s_3): &  \nonumber &\\
                                           & \text{Output } V(s_1, H(s_1, m), s_3) \nonumber
    \end{align}
    We can run this scheme multiple times to guarantee that the signer has knowledge of the 
    secret $d$ in the ZK scheme, allowing for this signature scheme to be secure.
\end{itemize}
\question{Signature Substitution Attacks} [20points]
\begin{itemize}
  \item[(a)] Chosen prefix attack has the birthday bound as a cryptographic hash
    function will compute $2^{n/2}$ messages as $H(P_1 || s_1) \ H(P_2 || s_2)$ is 
    equivalent to $H(m_1) = H(m_2)$ for sufficiently strong hash functions, therefore the 
    same birthday bounds apply. The chosen prefix attack can be used to obtain fraudulent 
    signatures. If a trusted CA signs a certificate, its signature can be used to sign another, 
    non-trusted domain using this CA as an attacker can exploit the fact that certificates are 
    Merkle D\aa mgard hash and calculate the $s_1, s_2$ needed to have a collision in the hash of 
    this new signed certificate, causing a forged certificate to be valid.
  \item[(b)] Files attached with same MD5: 1b79cb09eba5bcbfea522ac2e341c615
\end{itemize}

\question{Middleperson attacks} [20 points]
\begin{itemize}
  \item[(a)] Implementations should view all of the certificates down the chain from the 
    certificate with no CA attribute, as they are just signed by a possible attacker without CA
    qualifications. However, if an implementer failed to view these as invalid, Eve could become 
    a middleperson by having a certificate signed by the CA with no CA attribute
    (a valid certificate property), then use this certificate to sign other fraudulent domains or
    sign a false PK for Bob to give to Alice, amongst other attacks. The latter would allow Eve
    to impersonate Bob with the apparent added security of a signed public key.
  \item[(b)] The \textbf{Key Usage} (4.2.1.3) optional parameter could also be misinterpreted 
    by an implementation and lead to middleperson attacks. The standard states that this attribute
    \textit{MUST NOT be used to verify signatures on certificates or CRLs unless
    the corresponding keyCertSign or cRLSign bit is set.} However, in an implementation could 
    incorrectly assume that having the keyUsage bit is equivalent to having the CA bool set to 1,
    when in actuality, both must be set to $1$ in order for a certificate to have signing 
    privileges. This mistake could cause unwanted parties to forge certificates without having the
    CA bool set, carrying out attacks as mentioned in (a).
\end{itemize}

\question{Forged in the fires of CSELabs\dots} [20 points]
\begin{itemize}
  \item[(a)] Let us have messages with $k$ values of $k, k+1$. We then know that
    \begin{align}
      s_1k &= h_1 + \alpha r_1 \nonumber \\
      s_2k + s_2 &= h_2 + \alpha r_2 \nonumber
    \end{align}
    And since we know $h_1, h_2, s_1, s_2, r_1, r_2$ and want to find $k, \alpha$, we can simply 
    use linear algebra techniques, with a system of equations with 2 equations, 2 unknowns to 
    solve for $k, \alpha$, giving us Alice's secret key.
  \item[(c)] Even if an \textit{authentication} string is encrypted, an attacker that can see and
    modify messages can simply include the encrypted k-bit string in a forged message by capturing
    the messages before they reach the client / server and then replacing all but the k-bit 
    authenticator with the forged message, effectively becoming a middleperson.
  \item[(d)] If we were to use $H(m)$ rather than $m||c$, we could find a message that reveals all
    elements of the private key and thereby be able to make an existential forgery on all further
    signatures. However, this would be paramount to finding a value $m$ that hashes to $1^k$. But, 
    since we can find the hash on any message, due to the fact that $H$ is a random oracle, we can
    simply send in a message $m$, receive the signature, and then forge any message where the 
    indices of the $1$s values are a subset of the original message. For example, if our message
    had signature $110110110...110$, we could hash messages to find any message with $0$s in the 
    same places, thereby giving us a valid signature, as those private key windows are already 
    exposed. 
\end{itemize}
\question{Too Much (Private) Information} [20 points]
\begin{itemize}
  \item[(a)] Alice could build a PIR system with a public Hash $\textsf{H}: \{0, 1\}^* 
    \mapsto \{0, 1\}^l$ and implement a simple, multi-server PIR by having a client request 
    the following, given $r = $ index into the table.
    \begin{align}
      q_1 &= s \subseteq_R \{1, \cdots , r\}  \nonumber \\
      q_2 &= s \oplus \textsf{H}(cert)  \nonumber
    \end{align}
    To the servers and reconstruct $x_1 = r_1 \oplus r_2$ to get the boolean status of 
    whether or not the certificate is revoked, meaning $x_i = 1$ or $x_i = 0$. This system
    will sufficiently fit the needs of Alice's system, given a cryptographically secure hash.
  \item[(b)] For each server, the data $x_1, x_2, \cdots, x_r$ is a share of the data, generated
    by a Shamir Secret Sharing scheme. Let $x_i^j$ be the $j^{th}$ share of $x_i$. $P_j$ shares
    $x_i^j$ with all other servers, $P_j \mapsto P_k: S_{j, k}$. Then, $P_k$ creates 
    $S_k = \sum S_{j, k}$. It is then known that $S_k$ is a share of $S = \sum_j x_i^j = x_i$. 
    Thus, an user can choose $k$ servers out of the total $m$ that they \textit{trust} and 
    recreating $x_j$ from the shares.
\end{itemize}
\end{document}
