\documentclass[11pt]{article}
\usepackage{fullpage}
\usepackage{amsmath}
\usepackage{url}
\usepackage{hyperref}

\usepackage{amsfonts}
\newcounter{qnum}
\newcommand{\question}[1]{\stepcounter{qnum}\bigskip\noindent{\bf \arabic{qnum}. #1.}}

\newcommand{\E}{\mathop{\textrm{E}}}
\newcommand{\cM}{\mathcal{M}}
\newcommand{\cA}{\mathcal{A}}
\newcommand{\bbM}{\mathbb{M}}
\newcommand{\bbK}{\mathbb{K}}
\newcommand{\enc}{\textsf{Enc}}
\newcommand{\dec}{\textsf{Dec}}
\newcommand{\gen}{\textsf{Gen}}

\begin{document}
\begin{center}
{\Large \bf CSci 5471: Modern Cryptography}
\end{center}
{\bf Homework 2} \hfill {\bf due: March 7, 2019}
\medskip
\hrule
\medskip
\noindent{\bf Ground Rules.} You may choose to complete these
homeworks in a group of up to three students.  Each group should turn
in {\bf one} copy with the names of all group members on it.  You may
use any source you can find to help with this assignment but you {\bf
  must} explicitly reference any source you use besides the lecture
notes or textbook.  Electronically typeset copies of your solution
should be submitted to canvas by 11:59\textsc{pm} on the
date above.

\newcommand{\st}{\textsf{st}} \newcommand{\evolve}{\textsf{evolve}}
\newcommand{\fsec}[1]{\textsf{Fsec}_{#1}}
\question{Stream Ciphers, Pseudorandomness, and Distinguishers} [15 points] 
In practice, most variable output-length stream ciphers can be written
as a combination of two functions $\textsf{init}$ and
$\textsf{evolve}$.  $\textsf{init}$ takes as input a random seed $s$
and outputs a state $\st$.  $\evolve$ takes as input a state $\st$ and
outputs a pair $(\st',\textsf{nextWord})$ where $\st'$ is another
state and $\textsf{nextWord} \in \{0,1\}^w$ for some fixed word size
$w$.  \footnote{typical values for $w$ include $1$, $8$, $32$ and
  $64$.}  To generate the pseudorandom sequence $G(s,{\tt 1}^\ell)$ of
length $\ell$, the stream cipher sets $\st_0 = \textsf{init}(s)$,
computes $(\st_i,w_i) = \evolve(\st_{i-1})$ for
$i \in \{1,\dots, \ell/|w|\}$, and outputs
$w_1 || \cdots || w_{\ell/|w|}$.

\begin{itemize}
\item[(a)] [5 points] When used in synchronized mode, each party only needs to remember the
  current state $\st_i$ in order to construct the remaining bits of
  the stream. One question is whether such a scheme is ``forward
  secure:'' if an adversary manages to recover the current state, are
  messages already sent with the cipher still secure?  Define the
  forward security experiment $\fsec{\cA,G}(n)$ as follows:
\begin{enumerate}
\item $\ell \gets \cA({\tt 1}^n)$
\item Choose $s \in_R \{0,1\}^n$, set $\st_0 = \textsf{init}(s)$ and
  $(\st_i,w_i) = \evolve(\st_{i-1})$ for $i \in \{1,\dots\ell\}$.
\item Set $\sigma_0 = w_1||\cdots||w_\ell$, and choose $\sigma_1 \in_R
  \{0,1\}^{|\sigma_0|}$.
\item Choose $b \in_R \{0,1\}$ and let $b' \gets \cA({\tt 1}^n,
  \sigma_b, \st_\ell)$.
\item Output $1$ if $b=b'$ and $0$ otherwise.
\end{enumerate}
Say that a generator $G = (\textsf{init},\evolve)$ is forward secure
iff for every polynomial time $\cA$, there exists a negligible
$\varepsilon(n)$ such that $\Pr[\fsec{\cA,G}(n)=1] \leq \frac{1}{2} +
  \varepsilon(n)$. 

Prove that if $G$ is forward secure then it is also a variable
output-length pseudorandom generator.

\item[(b)] [5 points] A classical example of a statistical
  distinguisher $D$ comes from the Enigma machine used by the German
  military in the second World War: this machine had the property that
  no plaintext letter could encrypt to itself, which led in part to
  its eventual cryptanalytic defeat by the United Kingdom.  While the
  Enigma was not quite a stream cipher (the letters typed into the
  machine influenced its internal state), we could imagine a stream
  cipher $(\textsf{init},\evolve)$ with a similar property: namely,
  that for every state $\st$, the word computed by $\evolve(\st)$
  cannot be the zero word ${\tt 0}^w$.  Describe a distinguisher $D$
  for a stream cipher with the property, and compute its advantage for
  sequences of length $m\cdot 2^w$.

\item[(c)] [5 points] A more modern example of a stream cipher with
  statistical distinguisher attacks is the (Alleged) RC4 stream cipher
  (\url{https://en.wikipedia.org/wiki/RC4#Description}), which has
  word size $w=8$.  The known attacks mainly involve consecutive pairs
  of words, for example, the probability that two consecutive words output by
  RC4 are both ${\tt 0}$ is at least $2^{-16}(1+2^{-8})$.  Describe a
  distinguisher for the output of RC4 based on this fact, and compute
  the gap advantage of your distinguisher given a sequence of $m\cdot 2^{16}$ words.
\end{itemize}

\question{Balls and Bins and Random Mappings} [20 points]
\begin{itemize}
\item[(a)]  [5 points] Suppose we throw $m$ balls to $n$ bins. 
    What is the expected number of bins which have at least one ball? 
    Assume the bins are large enough to hold infinitely many balls and 
    that the throws are independent of each other.  (Hint: start with
    $m=1$; what is the probability that a specific bin is empty?
    Generalize to larger $m$ and use linearity of expectation plus the
    fact that if $k$ bins are empty, $n-k$ have at least one ball.)

\item[(b)]  [5 points] Suppose we have $\alpha \cdot n \ (0< \alpha \leq 1)$ balls and $n$ bins.
    After throwing all the balls into the bins randomly, what is the expected
    number of bins which have at least one ball?  

\item[(c)] [5 points]  Let $\mathcal{F}_{k,n}$ denote the set of all functions $f :
    \{0,1\}^k \rightarrow \{0,1\}^n$.  Prove that a random element of
    $\mathcal{F}_{k,2k+1}$ is one-to-one with probability at least
    $1/2$.  (Hint: what is the expected number of pairs $x \neq y$ such
    that $f(x) = f(y)$?)

\item[(d)] [5 points] Let $p$ be a prime, and let $x,y \in
  \mathbb{Z}_p$ be chosen arbitrarily.  Prove that for a randomly
  chosen degree $d < p$ polynomial $f \in \mathbb{Z}_p[x]$,
  $\Pr[f(x)\equiv y \bmod p] \leq d/p$.

\item[(e)]  Extra Credit: [5 points] In general, what is the
  probability that a random element of $\mathcal{F}_{k,n}$ is
  one-to-one when $k \leq n$?  You may use the approximation
  $(1-\frac{1}{n})^n \approx e^{-1}$.
\end{itemize}

\question{Block Ciphers, PRPs and PRFs} [20 points]
\begin{itemize}
\item[(a)] [10 points] Suppose we want to extend the key size of a Block
  Cipher $E : \{0,1\}^k \times \{0,1\}^\ell \rightarrow \{0,1\}^\ell$.
  One possible method would be to define $E' : \{0,1\}^{k+\ell} \times
  \{0,1\}^\ell \rightarrow \{0,1\}^\ell$ by $E'_{K||P}(x) =
  E_K(P\oplus x)$.  Prove that this does not improve the security of
  $E$ against brute force attacks: $E'$ can be completely broken with
  at most $2^k$ trial decryptions.  (Assume you have access to a moderate
  number of $(x,E'(x))$ pairs)

\item[(b)] [10 points] The Feistel construction can be seen as a
  method of constructing a secure block cipher on $2\ell$-bit blocks
  from a secure block cipher $\rho$ on $\ell$-bit blocks by repeating
  several rounds of the map $(L,R) \mapsto (R,L\oplus\rho(R))$.
  Consider instead the method of repeating several rounds of the map
  $(L,R) \mapsto (L\oplus\rho(L\oplus R), R \oplus \rho(L\oplus R))$.
  Prove that (i) this method is invertible, but (ii) even if we use
  truly random round functions $\rho_i$, the construction is not a
  secure PRP, for {\em any} number of rounds.

\item[(d)] [Extra Credit: 10 points] Suppose we are given a secure
  block cipher $E : \{0,1\}^k \times \{0,1\}^\ell \rightarrow
  \{0,1\}^\ell$ and want to construct a secure PRF $F : \mathbb{K}
  \times \{0,1\}^\ell \rightarrow \{0,1\}^{2\ell}$.  Give a
  construction of $F$ from $E$ that is provably secure.
\end{itemize}

\question{Block Cipher Modes} [20 points]  One of the common mistakes
that software developers make in their use of crypto is trying to
design their own block cipher modes of operation (Excluding the
developers that know not to use ECB mode).  Let's analyze a few
possibilities. 
\begin{itemize}
\item[(a)] [10 points] The classic book ``Applied Cryptography,'' by
  Bruce Schneier, lists several reasons ECB is not secure, one of
  which is that the same plaintext block always encrypts to the same
  ciphertext block.  One way to avoid this would be to incorporate a
  counter into every block, so that no two blocks are the same.
  Define the ECBC (ECB with counter) mode as follows:  Let $E$ be a
  block cipher on $\ell$-bit blocks, and define $\enc_K(m_1,\dots,m_t
  \in \{0,1\}^{t \times\ell/2}) = E_K(\langle 1 \rangle||m_1),
  E_K(\langle 2\rangle||m_2),\dots,E_K(\langle t\rangle||m_t)$, where
  $\langle i \rangle$ is the $\ell/2$ bit encoding of the integer
  $i$.  Prove that this scheme is not IND-CPA secure.  On the other
  hand, sketch a proof of security for up to $2^{\ell/8}$ message
  blocks if the counter is randomly selected for each message.

  (Notice that even when $\ell=128$ this is a weak security guarantee)

\item[(b)] [5 points]  Suppose that to reduce message expansion, the
  randomized counter from ECBC is XORed with the plaintext blocks
  rather than prepended, giving the scheme $\enc_K(m_1,\dots,m_t) =$
  ``Pick $R \leftarrow \{0,1\}^\ell$ and output $R,
  E_K(\langle R+1 \rangle \oplus m_1), E_K(\langle R+2\rangle \oplus
  m_2 ), \dots, E_K(\langle R+t \rangle \oplus m_t)$,''  Prove that
  this scheme is not IND-CPA secure.

\item[(c)] [5 points] On the other hand, let RXECB be defined by the
  following encryption algorithm:
\begin{tabbing}
$\enc_K(m_1,\dots,m_t)$ =\\
\ \ \=For $i \in \{1,\dots,t\}$, do:\\
\>\ \ \=Choose $r_i \in_R \{0,1\}^\ell$ \\
\>\> Set $p_i = E_K(r_i \oplus m_i)$.\\
Output: $r_1,p_1,r_2,p_2,\dots,r_t,p_t$.
\end{tabbing}
Prove that RXECB is IND-CPA secure (up to $2^{\ell/4}$ message blocks)
if $E$ is a PRP.

\item[(d)] [Extra Credit: 5 points] Another reason not to use ECB
  according to Schneier's book is that it does not protect against
  simple changes to messages, such as re-ordering or omitting the
  blocks.  For this reason, the book recommended CBC mode as an
  alternative.  Suppose we have a ciphertext $c^* = c_0,c_1,\dots,c_t$ that
  was encrypted using a secure block cipher in CBC mode.
  \begin{itemize}
  \item[(i)] Show how to compute the ciphertext for an arbitrary
    (block) suffix of the plaintext.
  \item[(ii)] Show how to compute a ciphertext that has an
    arbitrary bit flipped in the first plaintext block.
  \end{itemize}  
\end{itemize}

\question{Cryptanalysis: Un-chain my block} [25 points] Perhaps
because of Schneier's endorsement, or because it just seems more
complicated, CBC mode is a popular choice among fielded
cryptosystems.  Unfortunately, CBC can also be fairly easy to get
wrong: real implementations have turned out to be vulnerable to
several unfortunate attacks.
\begin{itemize}
\item[(a)] [10 points] The proof that CBC mode is IND-CPA secure
  relies on the use of random, unpredictable IVs.  However, in
  practice several schemes have used IVs that can be predicted in
  advance.  (For example, some implementations initially choose a
  random IV $R$ then use $R+1$, $R+2$, etc. for subsequent IVs; others
  use the last block of the previous ciphertext as the IV for the next
  message)

  Suppose we modify the IND-CPA game so that the challenger provides
  the adversary with the IV to be used {\em before} the adversary
  chooses the plaintext messages $m_0,m_1$. Show how to construct a
  distinguisher for this game.  (e.g., an IND-CPA adversary that can
  distinguish between $\enc_K(m_0)$ and $enc_K(m_1)$ in this game)

\item[(b)] [15 points] For this problem, you'll develop code that
  exploits predictable IVs to recover a secret message using chosen
  plaintexts.  Your program should interact with the following three
  scripts:
  \begin{itemize}
    \item \verb#/class/csci5471/hw2/nextIV# writes an 8-byte IV to the
      standard output and resets the plaintext buffer.
    \item \verb#/class/csci5471/hw2/buffer# reads up to 16 bytes from
      its standard input, and writes them to the plaintext
      buffer.\footnote{{\tt buffer} is destructive: calling a second
        time completely eradicates the old buffer}
    \item \verb#/class/csci5471/hw2/encrypt#  appends an 8-byte
      ``secret string'' to the plaintext buffer, pads the buffer
      length to a multiple of 8 with {\tt 0}-bytes, CBC encrypts the
      buffer (using Triple DES) with the current IV, and writes the ciphertext (including
      IV) to its standard output.  The IV and buffer are cleared.\footnote{if called with no IV, a random IV is chosen.} 
  \end{itemize}
  You can see in \verb#/class/csci5471/hw2/examples/iv.{c,py2,py3,java,ml}# examples of
  how to wrap these programs in function calls in some of the most
  popular ``favorite languages'' expressed on the self-assessment.
  (As in homework 1, each student will have a different, but fixed
  secret)

  Your canvas submission should include the secret string you found,
  and describe the algorithm you used at a high level.  The main
  submission should be accompanied by a source file (in your language
  of choice, provided that it will compile on a CSELabs Linux host)
  implementing your algorithm.

\item[(c)] [Extra Credit: 10 points] Another very common failure mode
  for CBC is the {\em padding oracle attack}.  The attack, originally
  due to Vaudenay, is described in
  \url{http://www.iacr.org/archive/eurocrypt2002/23320530/cbc02_e02d.pdf}~.
  Instances of this attack have proven to be pervasive, showing up in many
  standard networking protocols and libraries, and other surprising
  places such as popular web application frameworks.

  For this problem, you should develop an implementation of Vaudenay's
  padding attack.  Your program should interact with the following scripts:
  \begin{itemize}
  \item \verb#/class/csci5471/hw2/cbc# will pad (using PKCS\#7) an eight-byte
    ``secret message'' and write the CBC encryption (using Triple
    DES) of the message to the standard output.  (Each student gets
    a different, fixed ciphertext)
    
  \item \verb#/class/csci5471/hw2/pad_oracle# reads up to 24 bytes
    of ciphertext from its standard input, attempts to decrypt the
    bytes, and exits with status 0 if there is no padding error and
    status 1 otherwise.
  \end{itemize}
  \verb#/class/csci5471/hw2/examples/pad.{c,py,java}# give examples of how to
  wrap \verb#pad_oracle# as a function call.
  
  Your canvas submission for this problem should include the secret
  message, and a separate source file implementing the padding attack.
\end{itemize}


\end{document}
