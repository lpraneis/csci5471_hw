\documentclass[11pt]{article}
\usepackage{fullpage}
\usepackage{amsmath}
\usepackage{url}
\usepackage{hyperref}

\usepackage{amsfonts}
\newcounter{qnum}
\newcommand{\question}[1]{\stepcounter{qnum}\bigskip\noindent{\bf \arabic{qnum}. #1.}}

\newcommand{\E}{\mathop{\textrm{E}}}
\newcommand{\cM}{\mathcal{M}}
\newcommand{\cA}{\mathcal{A}}
\newcommand{\bbM}{\mathbb{M}}
\newcommand{\bbK}{\mathbb{K}}

\begin{document}
\begin{center}
{\Large \bf CSci 5471: Modern Cryptography}
\end{center}
{\bf Homework 1} \hfill {\bf due: February 21, 2019}
\medskip
\hrule
\medskip
\noindent{\bf Ground Rules.} You may choose to complete these
homeworks in a group of up to three students.  Each group should turn
in {\bf one} copy with the names of all group members on it.  You may
use any source you can find to help with this assignment but you {\bf
  must} explicitly reference any source you use besides the lecture
notes or textbook.  Electronically typeset copies of your solution
should be submitted to \url{canvas.cs5471.org} by 11:59\textsc{pm} on the
date above.

\question{Number Theory I}[25 points]
\begin{itemize}
  \item[(a)] Prove or disprove: $n^3 + (n+1)^3 + (n+2)^3$ is always
    mutiple of $9$.
  \item[(b)] Show that if $2^n-1$ is a prime, then $n$ is a prime.
  \item[(c)] Show that for all integers $n$, $4 \not| n^2 +2$.
  \item[(d)] To check whether a number is divisible by 11, sum the
    digits in the odd positions counting from the left (the first, third, ....)
    and then sum the remaining digits.  If the difference between the two sums
    is divisible by 11, then so is the original number. Otherwise it is not.
    Why does this work?
  \item[(e)] If $n>1$ is an integer and is not a prime, then show that
  there is a prime such that
  $$
  p ~|~ n \text{ and } p \leq \sqrt{n}.
  $$
 \item[(f)] [Extra credit: 5 points] Let $p$ be an odd prime. Show that if $a^h\equiv 1\mod p$ then
    $a^{ph} \equiv 1\mod p^2$.
\end{itemize}

\newpage

\question{Probability exercises} [25 points]
\begin{itemize}
\item[(a)] Let $X_1,\dots,X_n$ be independent random variables with
  uniform distributions on $\{0,1,\dots,k-1\}$.  What is the
  probability that there is no $i \in \{1,\dots,n-1\}$ such that $X_i =
  X_{i+1}$?

\item[(b)] Let $X$ be a non-negative real-valued random variable.
  Then {\em Markov's Inequality} states that $$\Pr[X > k \cdot \E[X]]
  \leq \frac{1}{k}\ .$$  Prove Markov's inequality.

\item[(c)] Let $A$ and $B$ be arbitrary events defined over the same
  experiment, and let $\bar{E}$ denote the complement of any event
  $E$.  Prove that $\Pr[A] \leq \Pr[A|B] + \Pr[\bar{B}]$.

\item[(d)] Prove the union bound: for any finite sequence of events
  $E_1, E_2, ... E_k$ defined over the same experiment,
  $\Pr[\bigcup_{i=1}^k E_i] \leq \sum_{i=1}^k \Pr[E_i]$.

\item[(e)] Let $X_1,\dots,X_n$ be as defined in part (a); let $Y_t =
  \# \{ i \mid X_i = t \}$ for each $t \in \{0,1,\dots,k-1\}$; and let $m >
  \lceil \frac{n}{k} +1 \rceil$.  Give a bound on the probability that
  for some $t$, $Y_t \geq m$.

\item[(f)] [Extra credit: 5 points] Give an exact expression for the
  probability in part (e).
\end{itemize}
\newpage
\question{Identification and Key Exchange Protocols} [25 points]
Alice, Bob, Carol, Dave and Eve want to build a chat server that allows
each of them to authenticate themselves to the server, $S$, using a shared secret
string, and then authenticate other users in chat sessions using
certificates and signatures.  They invent a set of protocols for these
goals:

\paragraph{Enrollment.}  In the enrollment phase, which only happens once:
\begin{enumerate}
\item The server picks a signing key pair $(VK_S,Sig_S)$ and announces
  $VK_S$ to all the users.

\item Each user $U$ picks a secret string $p_U$ and a signing key pair $(VK_U, Sig_U)$,
and gives $VK_U, p_U$ to the server.

\item The server gives each user $U$ a {\em ticket} $T_U =
  \mathsf{Sign}_S(H(p_U))$, where $H$ is a cryptographically secure
  hash function.

\item The server also gives each user $U$ a {\em certificate} $C_U =
  ID_U || VK_U || \mathsf{Sign}_S(ID_U||VK_U)$, where $ID_U$ is a
  unique (public) identifier string for user $U$ (e.g. ``Alice,''
  ``Bob,'' and so on).

\item Next the server picks a ``login key pair'' $(VK_{S1}, Sig_{S1})$
  and computes a ``server certificate'' $C_S =
  VK_{S1} || \mathsf{Sign}_S(VK_{S1})$.

\item Finally, the server locally erases $Sig_S$ as well as $p_U$ and
  $T_U$ for every user $U$.
\end{enumerate}
The certificates $C_I$ for $I \in \{A,B,C,D,E,S\}$ are assumed to be
public knowledge, but users are not required to store any certificates
other than their own.   (So each user $U$ stores $p_U$, $T_U$, $Sig_U$ and
$C_U$ but no other values).

\paragraph{Server Login.}  When user $A$ wants to initiate a
connection with the server:
\begin{enumerate}
\item $A \longrightarrow S : ID_A, N_A$, where $N_A$ is a fresh random
  $k$-bit string.
\item $S \longrightarrow A: C_S, \mathsf{Sign}_{S1}(ID_A || N_A)$.  $A$ checks that the
  signatures on $VK_{S1}$ and $ID_A || N_A$ are correct and stops if they
  are not.
\item $A \longrightarrow S: p_A, T_A$.  $S$ checks that $T_A$ is a valid
  sigature on $H(p_A)$, and if it is, accepts user $A$ with identity $ID_A$.
\end{enumerate}

\paragraph{Chat Authentication.} When users $A$ and $B$ want to
mutually authenticate each other:
\begin{enumerate}
\item $A \longrightarrow B: C_A, ID_B, N_A$, where $N_A$ is a fresh
  random $k$-bit string.  $B$ checks than $C_A$ is correctly signed
  (and that $ID_B$ matches his identity) and stops if it is not.
\item $B \longrightarrow A: C_B, N_B, \mathsf{Sign}_B(N_A)$, where
  $N_B$ is a fresh random $k$-bit string.  $A$ checks that $C_B$ is
  correctly signed, matches the identity $ID_B$ in step one, and that
  the signature on $N_A$ is correct.  If it is, she accepts $B$ with
  identity $ID_B$.
\item $A \longrightarrow B: \mathsf{Sign}_A(N_B)$.  $B$ checks that
  the message is a correct signature using $VK_A$ on the nonce in his
  previous message, and if it is, he accepts $A$ with identity $ID_A$.
\end{enumerate}
\paragraph{Questions:}
\begin{itemize}
\item[(a)] Describe an attack that will let Eve, an ordinary user, who knows only $T_E$,
  $VK_E$, $Sig_E$ and $VK_S$ (and the identifiers for Alice, Bob,
  Carol and Dave),  convince the server $S$ that she has
  any identity $ID_V$ for $V \in \{A,B,C,D\}$.  How would you modify
  the protocols to prevent this attack while still allowing the server
  to avoid storing tickets and the user to avoid signing at login?

\item[(b)] Describe an attack that will allow Eve to convince Bob that
  she is Alice by convincing Alice that she is Eve.  (That is, in your
  attack, Eve should initiate one chat with Bob and one with Alice,
  and at the end of the Chat Authentication protocols, Bob should have
  accepted Eve with ID ``Alice'' and Alice will have accepted Eve with
  ID ``Eve'').

\item[(c)] Describe an attack that will let Steve, the Starbucks
  eavesdropper who sees one authentication between Alice and $S$,
  impersonate Alice to the server in future authentications.

\item[(d)] Suppose that to defeat Steve, we modify the login protocol
  to encrypt all messages between user $U$ and the server using
  $h_2(p_U)$ as a symmetric key, where $h_2$ is a secure cryptographic hash
  function (different from the function $h$ used to form tickets).
  Assuming that the secrets $p_U$ are passwords chosen by the users,
  Steve may still be able to carry out a modified version of the
  attack in part (c).  Describe this attack.

\item[(e)] [Extra Credit: 10 points] What vaues known by $U$ and $S$
  after $U$'s registration protocol would be appropriate to use as
  input to $h_2$ to derive a symmetric key? Justify your answer.

\end{itemize}

\newpage
\newpage

\question{Cryptanalysis - The Two Time Pad} [25 points]
You take a consulting gig with the Lawyers Against Pirating Digital
Online Goods (LAPDOGs).  They are prosecuting a case in which they are
certain that some valuable digital goods have been stolen by nefarious
pirates.  The lawyers have confiscated the hard drive of one of these
pirates and found two encrypted, 1024-byte file on the drive.
Furthermore, the LAPDOGs have reason to believe that one of the files
contains an exact copy of a file from a highly valuable database owned
by one of their clients.  Forensic analysis suggests that this pirate
encrypted their files with one time pads.  Alas!

After further investigation, however, you find good evidence that the
pirate in question re-used the same pad twice to encrypt these
files. You can't believe your good luck!  Furthermore, the LAPDOGs
believe that one of the files is from the same client database as the
earlier case, but have not identified a client to bill for the second
ciphertext.

\begin{itemize}
\item[(a)] To see the ciphertexts, run the command
  \verb#/class/csci5471/hw1/make2tp# (on a CSELabs machine).  This program
  will write 2048 bytes of ciphertext on the standard output.  The
  first 1024 bytes are one file and the last 1024 are the other.  Both
  are encrypted with the same one-time pad.  Write a program that
  recovers the plaintexts corresponding to these two files.  You may
  assume that one of the plaintexts corresponds to a file from the
  client data base -  \verb#/class/csci5471/hw1/db/#  - and the other is in English.

  For the assignment, you should submit the two recovered plaintexts,
  and a description of the algorithm used to recover them.  Your
  submission should also be accompanied by a source file for your
  program.

  \begin{itemize}
  \item Each student will get a
    unique result from \verb#make2tp#, which you should
    redirect to a file.
  \item Suppose $c_1 = k \oplus f_i$ for some file $f_i$ in the
    database and $c_2 = k \oplus e$ for some english text file.
    Then what does $c_1\oplus c_2$ look like?
  \item The file \verb#/class/csci5471/hw1/ftable2.csv# contains a bigram frequency table
    for a sample of English text.  Each row starts with a character $first$;
    the value in the column $next$ is how often the character $next$
    follows $first$ in the sample.  The very first row gives the overall
    frequency of each character (blank space plus A-Z) in the sample.
  \end{itemize}
\item[(b)] [Extra Credit: 10 points] Develop a program to recover the
  plaintexts {\em without} using the suspected plaintext files in
  \verb#/class/csci5471/hw1/db/#: your program should work only on the assumption
  that both plaintexts are English text.  Run the command
  \verb#second2tp# to get a second pair of files for this question.

  Your solution for this portion should include the recovered
  plaintexts, and describe the algorithm used by your program. Your
  submission should be accompanied by a clearly labelled source file
  that implements this algorithm.
\end{itemize}

\end{document}
