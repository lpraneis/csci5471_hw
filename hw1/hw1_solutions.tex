\documentclass[11pt]{article}
\usepackage{fullpage}
\usepackage{amsmath}
\usepackage{amssymb}
\usepackage{url}
\usepackage{hyperref}

\usepackage{amsfonts}
\usepackage{fancyhdr}
\newcounter{qnum}
\newcommand{\question}[1]{\stepcounter{qnum}\bigskip\noindent{\bf \arabic{qnum}. #1.}}

\newcommand{\E}{\mathop{\textrm{E}}}
\newcommand{\cM}{\mathcal{M}}
\newcommand{\cA}{\mathcal{A}}
\newcommand{\bbM}{\mathbb{M}}
\newcommand{\bbK}{\mathbb{K}}
\newcommand\NetIDa{prane001}          
% \newcommand\NetIDb{collab?}          
% \newcommand\NetIDc{collab?}          

\begin{document}
\begin{center}
{\Large \bf CSci 5471: Modern Cryptography}
\end{center}
{\bf Homework 1} \hfill {\bf due: February 21, 2019}
\newline
{\bf \NetIDa} \hfill  
% {\bf \NetIDb} \hfill 
% \\
% {\bf \NetIDc} \hfill 

\medskip
\hrule
\medskip

\question{Number Theory I}
\begin{itemize}
  \item[(a)]   
    First, we are given the polynomial, $n^{3}  + (n+1)^{3} + (n+2)^{3}$, from which we can 
    simplify  to $3 x^{3} + 9 x^{2} + 15 x + 9$. In order to prove that each polynomial is
    divisible by 9, we are proving that $\forall x \in \mathbb{N},  \exists n \in \mathbb{N}$, 
    $9* n =  3 x^{3} + 9 x^{2} + 15 x + 9$. We can simplify this down to
    $3* n =   x^{3} + 3 x^{2} + 5 x + 3$. In order to prove that this holds, we have to prove 
    that $\forall x \in \mathbb{N},  3 \mid x^{3} + 3x^{2} + 5x + 3$. Proving that 
    $3 \mid 3x^{2} + 3$ is trivial, so we are left with proving $3 \mid x^{3} + 5x$. We can factor
    this to $3 \mid x(x^{2} + 5)$. Now, we are left with 2 cases to prove that this statement
    holds $\forall x \in \mathbb{N}$. 
    \newline \textit{Case 1:} $3\mid x$. Since $3\mid x$, $3 \mid x(x^{2} + 5)$ and so we are done.
    \newline \textit{Case 2:} $3\nmid x$. We can prove that $3\mid x^{2} + 5$ using 
    Fermat's Little Theorem. Because 3 is prime and $x^{2} = x^{3-1}$, we know 
    $x^{2} \equiv 1 \pmod{3}$, and thus, $(x^{2} + 5) \pmod{3} = 0$ as $6 \equiv 0 \pmod{3}$. 
    \newline Thus, we have proved that $\forall x \in \mathbb{N}$, $9 \mid n^{3}  + (n+1)^{3} +
    (n+2)^{3}$


  \item[(b)]
  \item[(c)]
  \item[(d)]
  \item[(e)] 
  \item[(f)] 
\end{itemize}

\newpage

\question{Probability exercises} 
\begin{itemize}
\item[(a)] 
\item[(b)]
\item[(c)] 
\item[(d)]
\item[(e)]
\item[(f)]
\end{itemize}
\newpage

\question{Identification and Key Exchange Protocols} 
\begin{itemize}
\item[(a)] 
\item[(b)]
\item[(c)]
\item[(d)]
\item[(e)]
\end{itemize}

\newpage

\question{Cryptanalysis - The Two Time Pad} 
\begin{itemize}
\item[(a)] 
\item[(b)] 
\end{itemize}

\end{document}
