\documentclass[11pt]{article}
\usepackage{fullpage}
\usepackage{amsmath}
\usepackage{amssymb}
\usepackage{url}
\usepackage{hyperref}

\usepackage{amsfonts}
\usepackage{fancyhdr}
\newcounter{qnum}
\newcommand{\question}[1]{\stepcounter{qnum}\bigskip\noindent{\bf \arabic{qnum}. #1.}}
\newcommand{\E}{\mathop{\textrm{E}}}
\newcommand{\cM}{\mathcal{M}}
\newcommand{\cA}{\mathcal{A}}
\newcommand{\bbM}{\mathbb{M}}
\newcommand{\bbK}{\mathbb{K}}
\newcommand\NetIDa{prane001}          
% \newcommand\NetIDb{collab?}          
% \newcommand\NetIDc{collab?}          

%Bibliography
\usepackage{biblatex}
\bibliography{References.bib}
% \addbibresource{References.bib}

\begin{document}
\begin{center}
{\Large \bf CSci 5471: Modern Cryptography}
\end{center}
{\bf Homework 1} \hfill {\bf due: February 21, 2019}
\newline
{\bf \NetIDa} \hfill  
% {\bf \NetIDb} \hfill 
% \\
% {\bf \NetIDc} \hfill 

\medskip
\hrule
\medskip

\question{Number Theory I}
\begin{itemize}
  \item[(a)]   
    First, we are given the polynomial, $n^{3}  + (n+1)^{3} + (n+2)^{3}$, from which we can 
    simplify  to $3 x^{3} + 9 x^{2} + 15 x + 9$. In order to prove that each polynomial is
    divisible by 9, we are proving that $\forall x \in \mathbb{N},  \exists n \in \mathbb{N}$, 
    $9* n =  3 x^{3} + 9 x^{2} + 15 x + 9$. We can simplify this down to
    $3* n =   x^{3} + 3 x^{2} + 5 x + 3$. In order to prove that this holds, we have to prove 
    that $\forall x \in \mathbb{N},  3 \mid x^{3} + 3x^{2} + 5x + 3$. Proving that 
    $3 \mid 3x^{2} + 3$ is trivial, so we are left with proving $3 \mid x^{3} + 5x$. We can factor
    this to $3 \mid x(x^{2} + 5)$. Now, we are left with 2 cases to prove that this statement
    holds $\forall x \in \mathbb{N}$. 
    \newline \textit{Case 1:} $3\mid x$. Since $3\mid x$, $3 \mid x(x^{2} + 5)$ and so we are done.
    \newline \textit{Case 2:} $3\nmid x$. We can prove that $3\mid x^{2} + 5$ using 
    Fermat's Little Theorem. Because 3 is prime and $x^{2} = x^{3-1}$ in addition to knowing
    that $x \not \equiv 0 \pmod{3}$, we know 
    $x^{2} \equiv 1 \pmod{3}$, and thus, $(x^{2} + 5) \pmod{3} = 0$ as $6 \equiv 0 \pmod{3}$. 
    \newline Thus, we have proved that $\forall x \in \mathbb{N}$, $9 \mid n^{3}  + (n+1)^{3} +
    (n+2)^{3}$
  \item[(b)]
    In order to prove that $2^{n} -1 \in \mathbb{P} \rightarrow n \in \mathbb{P}$, we are going to 
    prove the contrapositive: If $n$ is composite, then $2^{n} -1$ is composite. In order to prove
    this, let's first rewrite $2^{n} -1$ as follows: $2^{ab} -1$ for $a,b \in \mathbb{N}$. 
    We can then rewrite $2^{ab} -1$ as $(2^{a} -1) * (2^{a(b-1)} + 2^{a(b-2)} +
    \dots + 2^{a} + 1)$. If $n$ is composite, it implies that $\exists a \in \mathbb{N}$ where
    $a \neq n$ and $a \neq 1$. Thus, due to our factorization, this shows there exists a factor
    of $2^{ab} -1$ that is $(2^{a} -1) \not \in \{2^{ab} -1 , 1\}$, and therefore is composite. 
    By proving the contrapositive of this statement, we have proven the original statement.
  \item[(c)]
    Let us proceed with a proof by contradiction. Let us assume that $4 \mid (n^{2} + 2)$. This
    means that there exists a $x$ such that $4x = n^{2} +2$. Now, let us consider the
    cases: n is even or odd. 
    \newline \textit{n is even}: If n is even, we can express it as $n=2k$ for $k\in \mathbb{N}$
    This means that $4x = (2k)^{2} +2 \Rightarrow 4x = 4k^{2} +2$. If we divide by 2, this
    simplifies to $2x = 2k^{2} +1$, which makes the RHS odd and the LHS even.
    Thus, we have a contradiction.
    \newline \textit{n is odd}: If n is odd, we can express it as $n=2l +1$ for $l \in \mathbb{N}$.
    This means that $4x = (2l+1)^{2} +2 \Rightarrow 4x = 4l^{2} + 4l + 3$. We can rearrange this
    equation to yield the expression $4x  = 2(2l^{2} + 2l + 1) +1$. Again, the RHS is odd and the 
    LHS is even, marking a condition.
    \newline Because the even-odd parity is exhaustive for natural numbers, and in both cases
    we have proved a contradiction, we have proven that: $\forall n \in \mathbb{N}, 
    4 \not| n^{2} + 2$
  \item[(d)]
    We are set to prove that the following equivalence holds true: (odd $-$ even) $\equiv$ number 
    $\pmod{11}$. Let us consider an n-digit number such that $ N = a_{n} + 10a_{n-1} + 10^{2}
    a_{n-2} + \cdots + 10^{n-1}a_{1}$ where $\{ a_{1}, \cdots, a_{n}\}$ are the coefficients. If
    $ 11 | N$, then $N \equiv 0 \pmod{11}$ by the definition of modular equivalence. This means,
    by definition of $N$, that  $a_{n} + 10a_{n-1} + 10^{2}+  a_{n-2} + \cdots + 10^{n-1}a_{1} 
    \equiv 0 \pmod{11}$. Now, in order to prove the statement, we first introduce a Lemma that 
    $10^{x} \equiv (-1)^{x} \pmod {11}$ for any x. We can prove this quickly by considering that 
    $10$ and $-1$ are in the same equivalence class of $\mathbb{Z}_{11}$ as $11 | (10 + 1)$, thus 
    the class is closed under multiplication. Thus, we can rewrite $N$ as 
    $ N = a_{n} - a_{n-1} + a_{n-2} - a_{n-3} + \cdots - (-1)^{n-1}a_{1} \pmod{0}$ which is simply
    the representation of the even coefficients minus the odd coefficients.\cite{AOPS}
  \item[(e)] 
    If $n > 1$ is an integer and not prime, we must show that there is a prime such that:
    $p|n$ and $p \leq \sqrt{n}$. The first condition, that $p|n$, is given by the fundamental 
    theorem of arithmetic:
      For any number $>$ 1, n must itself be a prime number, thus $p|n$ or n must be expressed
      as the unique factorization of primes, in which case, there exits a $p$ such that $p|n$
    The second part, $p \leq \sqrt{n}$ is proved as follows. If $n$ is not prime and greater
    than 1, let it be expressed $n = xy$ where $1 < x \leq y$. Let $x$ be chosen such that 
    $p | x$. This means that $p \leq x \leq y$ and thus, $p^{2} \leq x^{2}\leq xy$.
    This means that $p \leq \sqrt{xy} = \sqrt{n}$ and thus both parts of the 
    statement are proven.
  \item[(f)] 
    We are given that $a^{h} \equiv 1 \pmod{p}$, which is equivalent to $p | (a^{h} -1)$.
    Thus, $\exists x: px = (a^{h} - 1) \rightarrow (px+1) = a^{h}$. If we raise both sides to 
    p, $(px+1)^{p} = a^{ph}$. Using binomial expansion, we can express this as the following:
    \[
      (px+1)^{p} = a^{ph} = 1 + xp^{2} + \sum_{k=2}^{p} \binom{p}{k} (xp)^{k}
    \]
    This equation shows, that every term other than $1$ is divisible by $p^{2}$. Thus, 
    $ p^{2} | (a^{ph} -1)$ and so, $a^{ph} \equiv 1 \pmod{p^{2}}$.
\end{itemize}

\newpage

\question{Probability exercises} 
\begin{itemize}
\item[(a)] 
\item[(b)]
\item[(c)] 
\item[(d)]
\item[(e)]
\item[(f)]
\end{itemize}
\newpage

\question{Identification and Key Exchange Protocols} 
\begin{itemize}
\item[(a)] 
\item[(b)]
\item[(c)]
\item[(d)]
\item[(e)]
\end{itemize}

\newpage

\question{Cryptanalysis - The Two Time Pad} 
\begin{itemize}
\item[(a)] 
\item[(b)] 
\end{itemize}

\pagebreak
% \printbibliography
\end{document}
