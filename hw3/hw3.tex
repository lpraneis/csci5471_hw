\documentclass[11pt]{article}
\usepackage{fullpage}
\usepackage{amsmath}
\usepackage{url}
\usepackage{hyperref}
\usepackage{graphicx}
\usepackage{amsfonts}
\newcounter{qnum}
\newcommand{\question}[1]{\stepcounter{qnum}\bigskip\noindent{\bf \arabic{qnum}. #1.}}

\newcommand{\E}{\mathop{\textrm{E}}}
\newcommand{\cM}{\mathcal{M}}
\newcommand{\cA}{\mathcal{A}}
\newcommand{\bbM}{\mathbb{M}}
\newcommand{\bbK}{\mathbb{K}}
\newcommand{\enc}{\textsf{Enc}}
\newcommand{\dec}{\textsf{Dec}}
\newcommand{\gen}{\textsf{Gen}}

\begin{document}
\begin{center}
{\Large \bf CSci 5471: Modern Cryptography}
\end{center}
{\bf Homework 3} \hfill {\bf due: March 27, 2019}
\medskip
\hrule
\medskip
\noindent{\bf Ground Rules.} You may choose to complete these
homeworks in a group of up to three students.  Each group should turn
in {\bf one} copy with the names of all group members on it.  You may
use any source you can find to help with this assignment but you {\bf
  must} explicitly reference any source you use besides the lecture
notes or textbook.  Electronically typeset copies of your solution
should be submitted on the course moodle by 11:59\textsc{pm} on the
date above.

\question{Broken MACs} [30 points]  Let $F : \{0,1\}^k \times
\{0,1\}^k \rightarrow \{0,1\}^k$ be a pseudorandom function.  Prove
that the following MAC constructions are insecure, even if $F$ is secure.
\begin{itemize}
\item[(a)] Define $\textsf{aMAC}$ on messages $m
  = m_1 || m_2 \in \{0,1\}^{2k}$  by $$\textsf{aMAC}_K(m_1||m_2) = F_K(m_1) ||
  F_K(F_K(m_2))\ .$$  
\item[(b)] Define $\textsf{bMac}$ on messages $m = m_1 || \cdots || m_\ell
  \in \{0,1\}^{\frac{k}{2} \times\ell}$ by $$\textsf{bMac}_K(m_1 || \cdots ||
  m_\ell) = r || \bigl( F_K(r) \oplus F_K(\langle 1 \rangle || m_1) \oplus
  \cdots \oplus F_K(\langle \ell \rangle || m_\ell) \bigr) \ ,$$ where $r \in_R
  \{0,1\}^k$ and $\langle i \rangle$ is the $k/2$-bit
  representation of integer $i$.
\item[(c)] Let $E$ be a block cipher modeled as an ideal cipher, and
  $H$ be a hash function modeled as a random oracle.  Show
  that the MAC $\textsf{cMac}$ defined by $\textsf{cMac}_K(m) =
  E_{H(m)}(K)$ is not secure.
\item[(d)] [Extra credit: 10 points] Prove, however, that
  $\textsf{dMac}$, defined by $\textsf{dMac}_K(m) = E_{H(m)}(K)\oplus
  K$ {\em is} secure.
\end{itemize}

\question{Broken Hashes} [20 points]
\begin{itemize}
\item[(a)] [10 points] Suppose we design a hash function (family) $H_K: \{0,1\}^* \rightarrow
  \{0,1\}^\ell$ from the block cipher $E : \{0,1\}^k \times
  \{0,1\}^\ell \rightarrow \{0,1\}^\ell$ by fixing a key $K \in
  \{0,1\}^k$ and setting $H(m) = \textsf{CBC-MAC}_K^E(m)$, that is, we
  divide $m$ into $\ell$-bit blocks $m_1,m_2,\dots,m_t$, set $c_0 =
  0^\ell$, $c_i = E_K(m_i \oplus c_{i-1})$, and let $H(m) = c_t$.  Show
  that this hash function (for known $K$) is not collision resistant.

\item[(b)] [5 points] Let $H_1, H_2 : \{0,1\}^{\ell+t} \rightarrow
  \{0,1\}^\ell$, where $t < \ell$, and define $H'(x) = H_1(x) || H_2(x)$.  Show that if
  $H_1$ or $H_2$ is collision-resistant, then so is $H'$.

\item[(c)] [5 points] However, show that even if $H_1$ is a truly
  random function, there exists a function $H_2$ such that $H'$ is not
  preimage-resistant: there is an algorithm that finds preimages to
  $H'$ in expected time $2^t$.
\end{itemize}

\question{MACs and Hashes together} [15 points]
\begin{itemize}
\item[(a)] [5 points] Let $H : \{0,1\}^* \rightarrow \{0,1\}^\ell$ be
  a collision resistant hash function (family).  Let $LB_t(x)$ return
  the last $t$ bits of $x$.  Prove that the hash function (family) $H'(x)
  = H(x) || LB_t(x)$ is collision resistant.

\item[(b)] [10 points] Let $H : \{0,1\}^* \rightarrow \{0,1\}^\ell$ be
  a collision-resistant hash function (family) and $M_K : \{0,1\}^n
  \rightarrow \{0,1\}^t$ be a (EUF-CMA) secure fixed-length MAC.
  Let $F_K : \{0,1\}^* \rightarrow \{0,1\}^\ell$ compute a tag
  $F_K(m)$ by dividing $m$ into $n$-bit blocks $m_1,\dots,m_\lambda$
  and computing $H(M_K(m_1),\dots,M_K(m_\lambda))$.  Prove that this
  composition is not generically secure, i.e. that there exist secure
  $H$ and $M$ such that $F_K$ is not EUF-CMA secure.

\item[(c)] [Extra credit: 10 points] Prove, however, that if the hash
  function $H$ in part (b) is modelled as a random oracle, then the
  resulting MAC $F_K$ is EUF-CMA secure.
\end{itemize}

\question{MACs and Encryption} [20 points] 
\begin{itemize}
\item[(a)] [10 points] Let $\enc$ be an IND-CPA secure encryption
  scheme and $M$ be a EUF-CMA secure MAC.  Define the composed
  encryption function $\enc^M_{K_1,K_2}(x) =\enc_{K_1}(x)||
  M_{K_2}(x)$ (``encrypt AND mac'').  Prove that there exists an
  IND-CPA secure encryption scheme $\enc$ and EUF-CMA secure mac $M$
  such that $\enc^M$ is not even IND-CPA secure.  

\item[(b)] [10 points] Since a block cipher in CBC mode can be used to
  build both an IND-CPA secure encryption scheme and a EUF-CMA secure
  block, a common mistake made in ``roll-your-own'' cryptosystems is
  to try to use the last ciphertext block to compute a MAC on the
  plaintext, e.g. to encrypt the message $m$, we compute the
  ciphertext $c = \textsf{CBCEncrypt}^E_K(m)$, let $c_\ast$ be the
  last block, compute tag $\tau = E_K(c_\ast)$.  and use $\langle c,\tau \rangle$ as an
  ``authenticated encryption'' scheme.  Prove that this scheme fails
  to provide chosen-ciphertext security.

\item[(c)] [Extra Credit: 10 points] In several encryption standards,
  ciphertexts may {\em optionally} be protected by a MAC.  The
  entire ciphertext is accompanied by metadata specifying information
  such as which keys and encryption algorithms to use; if a MAC is
  used the tag is computed over this ``associated data'' as well.
  Suppose that a ciphertext is encrypted using an implementation that
  is vulnerable to chosen ciphertext attack (such as the CBC padding
  attack), and a MAC is used to protect against this attack.  (i) Show
  how the ciphertext can still be attacked.  (ii) Assuming that the
  unauthenticated encryption option must still be supported, how would
  you design the authenticated encryption scheme to avoid this kind of
  attack?  Prove that your design is secure.
\end{itemize}

\question{Hash cycles} [15 points] For a given hash function $h$, a
hash chain starting from $x$ is recursively defined as follows:
  \begin{eqnarray*}
    H_0 &=& x \\
    H_i &=& h(H_{i-1}) \text{ for } i\geq 1.
  \end{eqnarray*}
  For the purpose of this homework, each student's hash function is the last k bits of
 SHA256  seeded with their CSELabs username. In other words,
  $h|_k(x) = \mathrm{SHA256}(\texttt{userid}||x) \bmod 2^k$.  (If you
  want to test your implementation of $h|_k$, you can use the program
  \verb#/class/csci5471/hw3/hashtest.py#.  \verb#hashtest.py k x# takes a
  number of bits $k$ and a string $x$ to hash on the command line and
  writes $h|_k(x)$ to the standard output; if $k$ is not a multiple
  of 8, the leading byte is zero-padded.

  After reading Section 2.1.6 of HAC (available from
  \url{http://www.cacr.math.uwaterloo.ca/hac/about/chap2.pdf}), answer the
  following questions.

\begin{itemize}
\item[(a)] [10 points] Write a program to compute the number of components,
    average/max tail length, and min/average/max cycle length in your
    $h|_{16}$. Your output should print out these 6 numbers.  Tail and
    cycle are defined in 2.35. To avoid the confusion, \emph{tail length} is
    defined as the number edges of the path to a cycle from a point.  In the
    following Figure, the number of components is 2, tail length of node 13 is 3,
    tail length of node 12 is 1, and the tail length of 6 is 0. The Average tail length
    is (3+2+1+1)/4 = 1.75. (i.e. average length of tails starting from the
    terminal points with no preimage.)  The cycle length of $[1, 4, 6, 9]$ is 4.

  \begin{figure}[h]
    \begin{center}
      \begin{tabular}{cc}
        \includegraphics[width=8cm]{functionalGraph} & 
        \includegraphics[width=6cm]{hashcollision} \\
        (a) & (b)
      \end{tabular}
      \caption{(a) Functional Graph (H.A.C. 2.1.6) and (b) Hash
        Collision Graph}
    \end{center}
  \end{figure}

  Your main submission should briefly describe how your program works
  and how to compile it; your canvas submission should include your
  source code.

\item[(b)] [5 points] Prove that the functional graph of any hash function with a fixed output
  length must have cycles.

\item[(c)] [Extra credit: 15 points] Find a cycle of $h|_k$ for the
  largest $k$ you can. To show that you found a cycle, present the
  initial value, and the number of times it needs to be hashed before
  it repeats. Under the assumption that your program runs correctly,
  here are the grading criteria. If $k>80$, you will get 15 points. If
  $72<k\leq 80$, you will get 12 points. If $64<k\leq 72$, you will
  get 9 points. If $56<k\leq 64$, you will get 6 points. If $32 <
  k\leq 56$, you will get 3 points.  For the purpose of verification,
  use your student ID as an initial hash value and provide the cycle
  length you found.  (Your main submission should include the cycle
  length, a brief description of your algorithm, and how to compile/run
  your source code on a CSELabs Linux machine.  Your canvas submission
  should include a separate source file for this question) 

\item[(d)] [Extra credit: 10 points] Find a (free) collision in
  $h|_k$: two distinct messages $m_1$, $m_2$ such that
  $h|_k(m_1)=h|_k(m_2)$.  For verification purposes, use your student ID as
  the initial hash value and provide the following values: the colliding
  hash (\emph{h4} in the figure), and its preimages in the tail and
  the cycle (\emph{h3} and \emph{h100} in figure (b), respectively).
  For $k>80$ this is worth 10 points, $72 < k \leq 80$ is worth 8 points, $64
  < k \leq 72$ is worth 6 points, $56 < k \leq 64$ is worth 4 points,
  and $32 < k \leq 56$ is worth 2 points.  Your main submission should
  include the collision values, a brief description of your algorithm, and how to compile/run
  your source code on a CSELabs Linux machine.  Your canvas submission
  should include a separate source file for this question.

  (Note: for (c) and (d), be careful not to use too much memory: a
  straightforward algorithm could use several terabytes for large
  enough $k$.  You need to use a more clever, space-efficient algorithm.
  Either design one yourself, or do some research for such an
  algorithm.)

  (Also Note: you are not expected to implement SHA256 for this
  problem.  You may use any standard crypto library (e.g. openssl,
  nacl, cryptography, pycrypto, bouncycastle,...) as long as your
  submission includes the appropriate build instructions.)

\end{itemize}

\end{document}
