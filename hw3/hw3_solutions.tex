\documentclass[11pt]{article}
\usepackage{fullpage}
\usepackage{amsmath}
\usepackage{url}
\usepackage{hyperref}
\usepackage{graphicx}
\usepackage{amsfonts}
\newcounter{qnum}
\newcommand{\question}[1]{\stepcounter{qnum}\bigskip\noindent{\bf \arabic{qnum}. #1.}}

\newcommand{\E}{\mathop{\textrm{E}}}
\newcommand{\cM}{\mathcal{M}}
\newcommand{\cA}{\mathcal{A}}
\newcommand{\bbM}{\mathbb{M}}
\newcommand{\bbK}{\mathbb{K}}
\newcommand{\enc}{\textsf{Enc}}
\newcommand{\dec}{\textsf{Dec}}
\newcommand{\gen}{\textsf{Gen}}
\newcommand\NetIDa{prane001}          

\begin{document}
\begin{center}
{\Large \bf CSci 5471: Modern Cryptography}
\end{center}
{\bf Homework 3} \hfill {\bf due: March 27, 2019}
\medskip
\\
{\bf \NetIDa} \hfill  
\hrule
\medskip
\noindent{\bf Ground Rules.} You may choose to complete these
homeworks in a group of up to three students.  Each group should turn
in {\bf one} copy with the names of all group members on it.  You may
use any source you can find to help with this assignment but you {\bf
  must} explicitly reference any source you use besides the lecture
notes or textbook.  Electronically typeset copies of your solution
should be submitted on the course moodle by 11:59\textsc{pm} on the
date above.

\question{Broken MACs} [30 points]
\begin{itemize}
\item[(a)] TODO
\item[(b)] TODO
\item[(c)] TODO
\item[(d)] TODO - EC
\end{itemize}

\question{Broken Hashes} [20 points]
\begin{itemize}
\item[(a)] TODO
\item[(b)] TODO
\item[(c)] TODO
\end{itemize}

\question{MACs and Hashes together} [15 points]
\begin{itemize}
\item[(a)] TODO 
\item[(b)] TODO 
\item[(c)] TODO - EC
\end{itemize}

\question{MACs and Encryption} [20 points] 
\begin{itemize}
\item[(a)] TODO
\item[(b)] TODO
\item[(c)] TODO - EC
\end{itemize}

\question{Hash cycles} [15 points]
\begin{itemize}
\item[(a)]  TODO
  \begin{itemize}
    \item max cycleLength:  195
    \item avg cycleLength:  45.22222222222222
    \item min cycleLength:  2
    \item max tailLength:  507
    \item avg tailLength:  169.88694353399993
    \item components:  9
  \end{itemize}
  Build Requirements: (python3, pip3, package networkx)
  \\
  This program works by modelling the hash function as a graph, and using various graph 
  algorithms to find the amount of cycles, the tail length, and the component breakdown.
  Takes ~2 minutes to find all of the necessary information. The terminal points are determined
  to be nodes with in-degree 0, and the tails are calculated as the distance from each terminal
  point to the closest point on the cycle in that component. 
  TODO - Finish explanaion
\item[(b)]  TODO
\item[(c)]  TODO - EC 
\item[(d)]  TODO - EC 

\end{itemize}

\end{document}
