\documentclass[11pt]{article}
\usepackage{fullpage}
\usepackage{amsmath}
\usepackage{url}
\usepackage{hyperref}
\usepackage{graphicx}
\usepackage{amsfonts}
\newcounter{qnum}
\newcommand{\question}[1]{\stepcounter{qnum}\bigskip\noindent{\bf \arabic{qnum}. #1.}}

\newcommand{\E}{\mathop{\textrm{E}}}
\newcommand{\cM}{\mathcal{M}}
\newcommand{\cA}{\mathcal{A}}
\newcommand{\bbM}{\mathbb{M}}
\newcommand{\bbK}{\mathbb{K}}
\newcommand{\enc}{\textsf{Enc}}
\newcommand{\dec}{\textsf{Dec}}
\newcommand{\gen}{\textsf{Gen}}
\newcommand\NetIDa{prane001}          

\begin{document}
\begin{center}
{\Large \bf CSci 5471: Modern Cryptography}
\end{center}
{\bf Homework 3} \hfill {\bf due: March 27, 2019}
\medskip
\\
{\bf \NetIDa} \hfill  
\hrule
\medskip
\question{Broken MACs} [30 points]
\begin{itemize}
\item[(a)] $\textsf{aMAC}$  is not secure as an adversary can create a new, valid MAC from
  existing queries to the $\textsf{aMAC}$ challenger. For example, given 
  $\textsf{aMAC}_k(m_1 || m_2)$ and $\textsf{aMAC}_k(m_3 || m_4)$, an adversary can create  a
  a valid $\textsf{aMAC}_k{m_1 || m_4}$ as it would be $ F_K(m_1) ||  F_K(F_K(m_4))\ .$ This 
  would be accepted by $\textsf{Ver}_k$, and thus would represent a valid forgery after only 
  2 queries to the oracle.
\item[(b)] $\textsf{bMAC}$ is insecure as with $\textsf{bMac}_K(m_1 || \cdots ||
  m_\ell) = r || \bigl( F_K(r) \oplus F_K(\langle 1 \rangle || m_1) \oplus
  \cdots \oplus F_K(\langle \ell \rangle || m_\ell) \bigr) \ ,$ we can forge a MAC by 
  considering $F_k(\langle x \rangle || m_x)$, where $\langle x \rangle$ is the first
  $\frac{k}{2}$ bits of $r$, gotten from our initial query to the MAC oracle and $m_x$ is the 
  second $\frac{k}{2}$ bits of $r$. We then replace r with $\langle x \rangle || m_x$, to get
  $$\textsf{bMac}_{K - forged} = \langle x \rangle || m_x || \bigl( F_K(\langle x \rangle || m_x)
    \oplus
    F_K(\langle 1 \rangle || m_1) \oplus \cdots \oplus F_K(r) \oplus \cdots \oplus 
  F_K(\langle \ell \rangle || m_\ell) \bigr) \ ,$$
  This is possible as $r$ is $k$-bits and if we fill m to the maximum $\ell$, we are guaranteed
  to have a value of $i$ for $m_i$ equal to the first $\frac{k}{2}$ bits of $r$. Since $\oplus$ is
  commutative, the tag will be the same but the message will be different, breaking the MAC.

  This can also be extended to choosing $r$ specifically with single-block messages. For example,
  if we set $r=\langle 1\rangle || m_0$ for some $m_0$, we can send the tag
  $t = \langle 1\rangle || m_0 || 0^k$ for any $m_0$ as $F_k(r) = F_k(\langle 1\rangle || m_0).$

\item[(c)] Because $H$ is modelled as a random oracle, we can query it for any message to
  get $H(m)$ and thus know the key for $E$. If the key for $E$ is known, and E is an
  ideal cipher, which is deterministic, an adversary can forge a $\textsf{dMAC}_k$ by 
querying $E_{H(m)}(.)$ with different values of the keyspace until the $E_{H(m)}(K) =$ 
tag on input. Then, with $K$ known, we can forge a tag on any message, $m$.
\end{itemize}

\question{Broken Hashes} [20 points]
\begin{itemize}
  \item[(a)] Because $E_k(M_{x-1} \oplus C_{x-2}) = C_{x-1}$, we can construct two messages, 
    $x_1, x_2$ such that $C_{x-1}$ for a given $x$ in $x_1$ is equal to $C_{x-1}$ for a given
    $x$ in $x_1$ as $E_k(M'_{x-1} \oplus C'_{x-2}) = E_k(M_{x-1} \oplus C_{x-2}) $. This is 
    possible as we can choose $m$ such that it produces the same ciphertext after being passed
    into $E_k^i$. Because of this, the hash function is not collision resistant.
\item[(b)] If we have a colliding message of $H$, it also collides in $H_1$ and $H_2$, as $H$ is 
  the concatenation of $H_1$ and $H_2$. Thus, if $H_1$ or $H_2$ is collision resistant, so will 
  be the corresponding part of $H$, making $H$ collision resistant to the same degree of security
  as the $H_b$ that is resistant.
\item[(c)] Even if $H_1$ is a TRF, there exists an $H_2$ such that $H'$ is not preimage resistant.
  We can prove this by defining an $H$ as follows:
  \begin{align}
    H'(x) & = H_1(x) || H_2(x) \\ \nonumber
    H_1(x) & = \textsf{TRF}\\ \nonumber
    H_2(x) & = H_1(x) \oplus \ell-bits(x)
  \end{align}
  There are $2^t$ possibilities for $x$ as we know $H_1(x)$ and $H_2(x)$ from $H(x)$. Thus, we
  simply have to guess the $t$ bits that we don't know, leaving us with $2^t$ enumerations.
\end{itemize}

\question{MACs and Hashes together} [15 points]
\begin{itemize}
  \item[(a)] Since $H$ is collision resistant, we can apply the conclusion of problem 2b to 
    conclude that $H'(x)$ is collision resistant.
  \item[(b)] Let us model $H$ as a Merkle-Damg\aa rd construction with $IV = 0^\ell.$ Given
    $m_1, \cdots , m_\lambda$, we need to prove that $F_k$ is not EUF-CMA secure, e.g, prove
    that $\Pr[Forge-F_k] \leq \textsf{negl}.$ We can do this by realizing that, for a CMA-Attcker,
    A, 
    $$ \Pr[Forge-F_k] \leq \textsf{Adv}(A) * \frac{\lambda^2}{2^\ell}$$
    $$\Rightarrow \Pr[Forge-F_k] \leq \frac{\lambda^2 *\varepsilon}{2^\ell}$$
    which is not necessarily negligible, proving that this construction, made from a EUF-CMA MAC
    and a collision resistant hash family, is not EUF-CMA secure.
  \item[(c)] If $H$ is modelled as a random oracle, there will be not be a collision unless 
    $M_k(m_i) = M_k(m_j)$ when $m_i \neq m_j$. However, if $M_k$ is EUF-CMA secure, 
    $\Pr[forgery] \leq \textsf{negl}$ and so $\Pr[M_k(m_i) = M_k(m_j)] \leq \textsf{negl}$. 
    Thus, the corresponding $\Pr[collision]$ and $\Pr[F_k-forged]$
    are both bound by $\textsf{negl}$,and so, $F_k$ is EUF-CMA secure.
\end{itemize}

\question{MACs and Encryption} [20 points] 
\begin{itemize}
  \item[(a)] Let $\textsf{Enc}$ be a CBC mode encryption scheme, which is known
    to be IND-CPA secure. Now, if we consider $\textsf{M}$ as an EUF-CMA secure MAC, we can
    construct another MAC, $\textsf{M'}$ such that $\textsf{M'} = b_0 || \textsf{M}$, where 
    $b_0$ is the first bit of $x$. We can show $\textsf{M'}$ is still EUF-CMA secure by asserting
    that prepending a bit to a secure MAC doesn't alter its forgability, and since $\textsf{M}$ is
    EUF-CMA secure, we can guarantee $\textsf{M'}$ to be EUF-CMA secure. Now, with this new MAC, 
    we can prove that $\enc^{\textsf{M'}}_{K_1,K_2}(x) =\enc_{K_1}(x)||  \textsf{M'}_{K_2}(x)$  
    is not IND-CPA secure as the first bit of the message is exposed in the MAC. Thus, an adversary
    can send two messages, $m_0, m_1$ where $b^{m_0}_0 \neq b^{m_1}_0$ and distinguish between them
    with probability 1.
% \item[(b)]  We can break this scheme by asking the oracle for the decryption / encryption
%   of several chosen messages. First, $m^0 = m^0_0 || m^0_1 || 1^B || 0^B$ where $B$ is the 
%   block length. This will give us $c^0 = IV || \enc(IV \oplus m^0_0) || \cdots || 
%   \enc(1^B) || \enc(\enc(1^B)), t = \enc(\enc(\enc(1^B))). $ % TODO - Finish 4b
\item[(b)] Because CBC-MAC requires that the IV not be random, we can exploit the IV 
for this $\textsf{CBCEncrypt}$ scheme to prove it is not chosen-ciphertext secure. Given a valid
$\ell-$block message, $m^0$ and ciphertext, tag pair: $\langle c^0, \tau^0 \rangle$,
we know the IV used as $c^0_0$ and the input to the block cipher as $c^0_\ell$. Thus, we can 
re-use this same tag, $\tau^0$ for any message that ends with the $c^0_\ell$, given that is 
also a $\ell-$block message. An attacker can exploit the \textsf{Enc}(.) and \textsf{Dec}(.)
oracles to construct a message with this ending byte, while re-using the tag from the original
message. Thus the adversary can distinguish messages with probability 1 and forge an arbitrary
message with advantage 1.
\item[(c)]
  \begin{itemize}
    \item[(i)]
      The ciphertext can still be attacked in a MAC-then-encrypt mode as 
      $$C = \textsf{Enc}(m||\textsf{HMAC}(m))$$
      When C is decrypted to get $m||\textsf{HMAC}(m)$, the decryption algorithm must work
      out the padding for $m$, and thus is vulnerable to padding oracle attacks.
    \item[(ii)]
      Encrypt-then-MAC mode still supports unauthenticated encryption as the algorithm
      first encrypts the message and then (optionally) computes a MAC.  This is secure to 
      ciphertext attacks as the receiver first validates the MAC before decrypting the message. 
      This means that an attacker cannot exploit padding attacks on the MAC.

  \end{itemize}

\end{itemize}

\question{Hash cycles} [15 points]
\begin{itemize}
\item[(a)]
  \begin{itemize}
    \item max cycleLength:  195
    \item avg cycleLength:  45.22222222222222
    \item min cycleLength:  2
    \item max tailLength:  507
    \item avg tailLength:  169.88694353399993
    \item components:  9
  \end{itemize}
  Build Requirements: (python3, pip3, package networkx)
  \\
  This program works by modelling the hash function as a graph, and using various graph 
  algorithms to find the amount of cycles, the tail length, and the component breakdown.
  Takes ~2 minutes to find all of the necessary information. The terminal points are determined
  to be nodes with in-degree 0, and the tails are calculated as the distance from each terminal
  point to the closest point on the cycle in that component. The other metrics are calculated
  as averages as the data from the terminal points.
\item[(b)] Since every value in $\{0, 1\}^*$ has a hashed value, we can consider the hash 
  function as a graph out nodes with out-degree 1. For a $n-node$ graph, we have $n$ edges
  as every function has at most $1$ in-degree and $1$ out-degree. Thus, by the definition of
  a cycle in a directed graph, we are required to have at least one cycle, as the maximum number
  of edges in such a graph (without cycles) would be $n-1$.

\end{itemize}

\end{document}
